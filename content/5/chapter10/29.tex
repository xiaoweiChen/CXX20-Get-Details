In computer science, a thread of execution is the smallest sequence of programmed instructions that a scheduler can manage independently that is typically a part of the operating system. The implementation of threads and processes differs between operating systems, but in most cases, a thread is a process component. Multiple threads can exist within one process, executing concurrently and sharing resources such as memory, while different processes do not share these resources. For the details, read the Wikipedia article about \href{https://en.wikipedia.org/wiki/Thread_(computing)}{threads}.