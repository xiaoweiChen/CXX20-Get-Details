\begin{center}
\includegraphics[width=0.4\textwidth]{content/3/chapter4/images/40.png}\\
Cippi正在使用她的新工具
\end{center}

C++20的改进使模板更加一致,在编写泛型程序时更不容易出错。

\subsubsubsection{4.6.1\hspace{0.2cm}条件显式构造函数}

有时需要一个类,需要有接受不同类型的构造函数。例如,一个VariantWrapper类,包含接受各种类型的std::variant。

\hspace*{\fill} \\ %插入空行
\noindent
\textbf{包含std::variant的VariantWrapper类}
\begin{lstlisting}[style=styleCXX]
class VariantWrapper {
	std::variant<bool, char, int, double, float, std::string> myVariant;
};
\end{lstlisting}

可以使用bool、char、int、double、float或std::string初始化VariantWrapper,VariantWrapper类需要为每个列出的类型提供相应构造函数。懒惰是一种美德——至少对程序员来说是这样——因此,可以使用泛型构造函数。

Implicit类展示了一个泛型构造函数。

\hspace*{\fill} \\ %插入空行
\noindent
\textbf{泛型构造函数}
\begin{lstlisting}[style=styleCXX]
// implicitExplicitGenericConstructor.cpp

#include <iostream>
#include <string>

struct Implicit {
template <typename T>
	Implicit(T t) {
		std::cout << t << '\n';
	}
};

struct Explicit {
template <typename T>
	explicit Explicit(T t) {
		std::cout << t << '\n';
	}
};

int main() {
	
	std::cout << '\n';
	
	Implicit imp1 = "implicit";
	Implicit imp2("explicit");
	Implicit imp3 = 1998;
	Implicit imp4(1998);
	
	std::cout << '\n';
	
	// Explicit exp1 = "implicit";
	Explicit exp2{"explicit"};
	// Explicit exp3 = 2011;
	Explicit exp4{2011};
	
	std::cout << '\n';

}
\end{lstlisting}

现在,问题来了。泛型构造函数(第7行)是一个全能构造函数,可以接受任何类型。但构造函数太贪婪了,可以通过在构造函数(第14行)前面放置explicit关键字,避免隐式转换(第31和33行),只有显式调用(第32和34行)的方式是有效的。

\begin{tcblisting}{commandshell={}}
implicit
implicit
1998
1998

explicit
2011
\end{tcblisting}

\begin{center}
隐式和现实泛型构造函数
\end{center}

C++20中,explicit更有用。假设有一个MyBool类型,只支持从bool类型的隐式转换,而不支持其他隐式转换。这种情况下,explicit可以有条件地使用。

\hspace*{\fill} \\ %插入空行
\noindent
\textbf{允许从bool类型进行隐式转换的泛型构造函数}
\begin{lstlisting}[style=styleCXX]
// conditionallyConstructor.cpp

#include <iostream>
#include <type_traits>
#include <typeinfo>

struct MyBool {
	template <typename T>
	explicit(!std::is_same<T, bool>::value) MyBool(T t) {
		std::cout << typeid(t).name() << '\n';
	}
};

void needBool(MyBool b){ }

int main() {

	MyBool myBool1(true);
	MyBool myBool2 = false;
	
	needBool(myBool1);
	needBool(true);
	// needBool(5);
	// needBool("true");

}
\end{lstlisting}

显式(!std::is\_same<T, bool>::value)表达式保证MyBool只能由布尔值隐式创建。函数std::is\_same是来自\href{https://en.cppreference.com/w/cpp/header/type_traits}{类型特性库}的编译时谓词。因为是编译时谓词,所以std::is\_same可在编译时进行计算,并返回一个布尔值。因此,可以对布尔类型进行隐式转换(第19和22行),但不能从int和C-string类型进行转换(第23和24行,注释行)。

\subsubsubsection{4.6.2\hspace{0.2cm}非类型模板参数}

C++支持非类型作为模板参数。本质上非类型可能是

\begin{itemize}
\item 
整数和枚举数

\item 
指向对象、函数和类属性的指针或引用

\item 
std::nullptr\_t
\end{itemize}

\begin{tcolorbox}[breakable,enhanced jigsaw,colback=mygreen!5!white,colframe=mygreen!75!black,title={典型的非类型模板参数}]
	
当我问我班上的学生是否使用过非类型作为模板形参时,他们说:没有!

那只能由我回答这个棘手的问题了,并展示一个使用的非类型模板参数示例:

\hspace*{\fill} \\ %插入空行
\noindent
\textbf{定义std::array}
\begin{lstlisting}[style=styleCXX]
std::array<int, 5> myVec;
\end{lstlisting}

常量5是一个非类型的模板参数。
\end{tcolorbox}	

从C++98开始,C++社区就一直在讨论支持浮点模板形参。现在,C++20支持浮点数、文字类型和字符串文字作为非类型。

\noindent
\textbf{4.6.2.1\hspace{0.2cm}浮点和文字类型}

文字类型有以下两个属性:

\begin{itemize}
\item 
所有基类和非静态数据成员都是public,且不可变

\item 
所有基类和非静态数据成员的类型都是结构类型或数组
\end{itemize}

文字类型必须具有constexpr构造函数。下面的程序使用浮点类型和文字类型作为非类型模板参数:

\hspace*{\fill} \\ %插入空行
\noindent
\textbf{使用浮点和文字类型作为非类型模板参数}
\begin{lstlisting}[style=styleCXX]
// nonTypeTemplateParameter.cpp

struct ClassType {
	constexpr ClassType(int) {}
};

template <ClassType cl>
auto getClassType() {
	return cl;
}

template <double d>
auto getDouble() {
	return d;
}

int main() {

	auto c1 = getClassType<ClassType(2020)>();
	
	auto d1 = getDouble<5.5>();
	auto d2 = getDouble<6.5>();

}
\end{lstlisting}

ClassType有一个constexpr构造函数(第4行),可以用作模板参数(第19行)。函数模板getDouble(第13行)也是如此,只能接受double类型。我想强调的是,每次调用函数模板getDouble(第21和22行)都会创建一个新函数getDouble,这个函数是给定double值的全特化版本。

C++20起,字符串可以用作非类型模板形参。

\hspace*{\fill} \\ %插入空行
\noindent
\textbf{4.6.2.2\hspace{0.2cm}字符串字面值}

StringLiteral类有一个constexpr构造函数。

\begin{lstlisting}[style=styleCXX]
// nonTypeTemplateParameterString.cpp

#include <algorithm>
#include <iostream>

template <int N>
class StringLiteral {
public:
	constexpr StringLiteral(char const (&str)[N]) {
		std::copy(str, str + N, data);
	}
	char data[N];
};

template <StringLiteral str>
class ClassTemplate {};

template <StringLiteral str>
void FunctionTemplate() {
	std::cout << str.data << '\n';
}

 int main() {
	
	std::cout << '\n';
	
	ClassTemplate<"string literal"> cls;
	FunctionTemplate<"string literal">();
	
	std::cout << '\n';

}
\end{lstlisting}

StringLiteral是一个文字类型,因此可以用作ClassTemplate(第15行)和FunctionTemplate(第18行)的非类型模板参数。constexpr构造函数(第9行)接受C-string作为参数。

\begin{tcblisting}{commandshell={}}
string literal
\end{tcblisting}

读者们可能想知道,为什么需要字符串字面量作为非类型模板参数呢?

\begin{tcolorbox}[breakable,enhanced jigsaw,colback=mygreen!5!white,colframe=mygreen!75!black,title={编译时的正则表达式}]

字符串字面量的一个经典用例就是\href{https://github.com/hanickadot/compile-time-regular-expressions}{编译时正则表达式解析}。C++23已经有了一个提案:\href{http://www.open-std.org/jtc1/sc22/wg21/docs/papers/2019/p1433r0.pdf}{P1433R0:编译时正则表达式}。Hana Dusíková作为提案的作者,希望C++在编译时可以使用正则表达式:“当前的std::regex设计和实现[\href{https://en.cppreference.com/w/cpp/regex}{正则表达式库}]太慢了,主要是因为RE[正则表达式]模式是在运行时解析和编译的。用户通常不需要运行时RE[正则表达式]解析器引擎。许多常见的用例中,模式是在编译时就已知了。我认为这违背了C++的承诺:『不使用就不支付』”。

\hspace*{\fill} \\ %插入空行
若RE[正则表达式]在编译时已知,则应该在编译期间检查。但std::regex的设计不允许这种[编译时求值]的方式,因为RE输入是一个运行时字符串,语法错误会作为异常抛出。

\end{tcolorbox}

\begin{tcolorbox}[breakable,enhanced jigsaw,colback=mygreen!5!white,colframe=mygreen!75!black,title={总结}]
\begin{itemize}
\item 
有条件显式构造函数,允许显式控制泛型构造函数中可以使用的类型。

\item 
C++20支持更多类型作为非类型模板参数:浮点数和字符串字面值。
\end{itemize}
\end{tcolorbox}

\newpage