
\begin{center}
\includegraphics[width=0.7\textwidth]{content/3/chapter4/images/2.png}\\
\end{center}

为了充分理解概念,需要先了解概念出现的原因。

\subsubsubsection{4.1.1\hspace{0.2cm}两种错误的方法}

C++20之前,可以有两种截然相反的方式来思考函数或类:为特定类型或泛型类型定义。后者,称之为函数模板或类模板。不过,这两种方法都有各自的问题:

\hspace*{\fill} \\ %插入空行
\noindent
\textbf{4.1.1.1\hspace{0.2cm}太具体}

为每种类型的重载函数,或是重新进行类实现是一项乏味的工作。为了避免这种重复,通常会使用类型转换。不过,这种方式看似拯救,却又是诅咒。

\hspace*{\fill} \\ %插入空行
\noindent
\textbf{类型的隐式转换}
\begin{lstlisting}[style=styleCXX]
// tooSpecific.cpp

#include <iostream>

void needInt(int i){
	std::cout << "int: " << i << '\n';
}

int main(){

	std::cout << std::boolalpha << '\n';

	double d{1.234};
	std::cout << "double: " << d << '\n';
	needInt(d);

	std::cout << '\n';

	bool b{true};
	std::cout << "bool: " << b << '\n';
	needInt(b);

	std::cout << '\n';

}
\end{lstlisting}

例子(第13行)以double开始,以int结束(第15行)。第二次,以bool开始(第19行),并以int结束(第21行)。

\begin{center}
\includegraphics[width=0.6\textwidth]{content/3/chapter4/images/3.png}\\
\end{center}

例子进行了两次类型隐式转换。

\hspace*{\fill} \\ %插入空行
\noindent
\textbf{4.1.1.1.1\hspace{0.2cm}窄化转换}

使用double类型调用getInt(int a)可以实现窄化转换。窄化转换是一种类型转换,包括精度损失。但有时,开发者并不想有任何损失。

\hspace*{\fill} \\ %插入空行
\noindent
\textbf{4.1.1.1.2\hspace{0.2cm}类型提升}

反过来会更好吗?

使用bool类型调用getInt(int a),可将bool类型提升为int类型。惊讶吗?许多C++开发人员在添加两个bool时,并不知道他们会得到的是哪种数据类型。

\hspace*{\fill} \\ %插入空行
\noindent
\textbf{对两个布尔值进行加法运算}
\begin{lstlisting}[style=styleCXX]
template <typename T>
auto add(T first, T second){
	return first + second;
}

int main(){
	add(true, false);
}
\end{lstlisting}

\href{https://cppinsights.io/s/9bd14f99}{C++ Insights}在编译器在实例化中转换函数模板后,并将上面的源代码可视化。

\begin{center}
\includegraphics[width=0.8\textwidth]{content/3/chapter4/images/1-1.png}\\
bool提升为int
\end{center}

第8-14行是\href{https://cppinsights.io/}{C++ Insights}截图中的关键行。函数模板add的模板实例化,使用返回类型int创建一个全特化函数,从而两个bool类型的参数都隐式提升为int。

因为我们不想为每种类型重载函数或重新实现类,来尝试依赖于转换的魔力。

让我试试另一种方式,用一个泛型函数。

\hspace*{\fill} \\ %插入空行
\noindent
\textbf{4.1.1.2\hspace{0.2cm}太通用}

容器排序是一个很普遍的需求。若容器的元素支持排序,那么就应该适用于每个容器。下面的示例中,我将标准算法std::sort应用于标准容器std::list。

\hspace*{\fill} \\ %插入空行
\noindent
\textbf{对std::list排序}
\begin{lstlisting}[style=styleCXX]
// tooGeneric.cpp

#include <algorithm>
#include <list>

int main(){

	std::list<int> myList{1, 10, 3, 2, 5};

	std::sort(myList.begin(), myList.end());

}
\end{lstlisting}

\begin{center}
\includegraphics[width=1.0\textwidth]{content/3/chapter4/images/4.png}\\
对std::list进行排序时,编译器报错
\end{center}

我不想解析这么长的信息,但还是要看看出什么问题了。看一下本例中使用的\href{https://en.cppreference.com/w/cpp/algorithm/sort}{std::sort}特定重载的签名。

\hspace*{\fill} \\ %插入空行
\begin{lstlisting}[style=styleCXX]
template< class RandomIt >
constexpr void sort( RandomIt first, RandomIt last );
\end{lstlisting}

std::sort使用了奇怪的参数类型RandomIt。RandomIt代表随机访问迭代器,也就是编译错误提供的关键信息。std::list只提供双向迭代器,但std:sort需要一个随机访问迭代器。下图显示了为什么std::list不支持随机访问迭代器。

\begin{center}
\includegraphics[width=0.8\textwidth]{content/3/chapter4/images/5.png}\\
\end{center}

去看一下cppreference.com上的std::sort文档,会发现一些模板参数对类型有要求。它们对类型提出了概念性的要求,这些要求形成了C++20的概念。

\hspace*{\fill} \\ %插入空行
\noindent
\textbf{4.1.1.3\hspace{0.2cm}概念}

概念对模板参数施加语义约束,std::sort具有接受比较器的重载。

\begin{lstlisting}[style=styleCXX]
template< class RandomIt, class Compare >
constexpr void sort(RandomIt first, RandomIt last, Compare comp);
\end{lstlisting}

下面是std::sort重载的类型要求:

\begin{itemize}
\item
RandomIt必须满足ValueSwappable和LegacyRandomAccessIterator的要求

\item
解引用RandomIt的类型必须满足MoveAssignable和MoveConstructible的要求。

\item
可解引用的RandomIt的类型必须满足Compare的要求。
\end{itemize}

ValueSwappable或LegacyRandomAccessIterator之类的要求就是类型需求,其中一些需求在C++20的\href{https://en.cppreference.com/w/cpp/language/constraints}{概念}中形式化了。

并且,std::sort还需要一个LegacyRandomAccessIterator,在C++20中称为random\_access\_iterator(是<iterator>的一部分):

\hspace*{\fill} \\ %插入空行
\noindent
\textbf{std::random\_access\_iterator}
\begin{lstlisting}[style=styleCXX]
template<class I>
	concept random_access_iterator =
		bidirectional_iterator<I> &&
		derived_from<ITER_CONCEPT(I), random_access_iterator_tag> &&
		totally_ordered<I> &&
		sized_sentinel_for<I, I> &&
		requires(I i, const I j, const iter_difference_t<I> n) {
			{ i += n } -> same_as<I&>;
			{ j + n } -> same_as<I>;
			{ n + j } -> same_as<I>;
			{ i -= n } -> same_as<I&>;
			{ j - n } -> same_as<I>;
			{ j[n] } -> same_as<iter_reference_t<I>>;
		};
\end{lstlisting}

类型I支持概念random\_access\_iterator,若支持概念bidirectional\_iterator和以下所有需求。例如,\{i += n\} \texttt{->} same\_as<I\&>作为需求表达式的一部分,意味着对于类型为i的值,\{i += n\}是一个有效的表达式,返回类型为I\&。对于链表,std::list支持的是bidirectional\_iterator,而非std::sort要求的random\_access\_iterator。

现在需要random\_access\_iterator的算法,接收到birectional\_iterator时,就会显示一条简洁易懂的错误消息,明确的表示迭代器不满足random\_access\_iterator的要求。

\begin{center}
\includegraphics[width=0.7\textwidth]{content/3/chapter4/images/6.png}\\
标准模板库
\end{center}

\begin{tcolorbox}[breakable,enhanced jigsaw,colback=blue!5!white,colframe=blue!75!black,title={泛型编程的本质}]
我想引用Alexander Stepanov(标准模板库的创建者)和Daniel Rose(信息检索研究员)写的书\href{https://www.fm2gp.com/}{《From Mathematics to Generic Programming》}中的一段话来开始这段简短的回顾:“泛型编程的本质在于概念的思想,概念是描述一系列相关对象类型的一种方式。”这些相关的对象类型可以是整型,如bool、char或int。概念是对相关类型的一组需求,例如:所支持的操作、语义,以及时间和空间的复杂度。

\hspace*{\fill} \\ %插入空行
标准模板库(STL)作为一个基于概念的通用库,由三个部分组成:容器,运行在容器上的算法,以及连接它们的迭代器。

\hspace*{\fill} \\ %插入空行
每个容器都提供了符合其结构的迭代器,算法可以对这些迭代器进行操作。容器(如序列容器或关联容器)基于半开放范围模型。迭代器提供对容器元素的访问,并且可以遍历容器,从而可对其进行比较。STL的抽象基于半开放范围模型和迭代器等概念,从而可以使用STL的容器和算法。
\end{tcolorbox}

那么,概念的优势是什么?

\subsubsubsection{4.1.2\hspace{0.2cm}概念的优势}

\begin{itemize}
\item
模板参数的需求描述是接口声明的一部分。

\item
函数的重载和类模板的特化可以基于概念。

\item
概念可用于函数模板、类模板和类或类模板的泛型成员函数。

\item
因为编译器会将模板形参的要求与给定的模板实参进行比较,所以可得到明确的错误消息。

\item
可以使用预定义的概念,也可以自定义概念。

\item
auto和概念的用法统一,可以用概念来代替auto。

\item
若函数声明使用了概念,则它自动成为函数模板。因此,编写函数模板就会如同编写函数一样简单。
\end{itemize}


\subsubsubsection{4.1.3\hspace{0.2cm}漫长的历史}

我第一次听说概念是在2005 - 2006年左右,这让我想起Haskell的课程,Haskell类型就具有类似的接口。下面是\href{https://en.wikipedia.org/wiki/Haskell_(programming_language)}{Haskell的}类型类层次结构的一部分。

\begin{center}
\includegraphics[width=0.8\textwidth]{content/3/chapter4/images/7.png}\\
Haskell类型的层次结构
\end{center}

但C++的概念不同。以下是我观察到的结果。

\begin{itemize}
\item
Haskell中,类型都必须是实例。C++20中,类型必须满足概念的需求。

\item
C++中,概念可以用在模板的非类型参数上。例如,像值5这样的数字就是非类型参数。例如,当想要一个包含5个元素的整型数组std::array时,可以使用非类型参数5:std::array<int, 5> myArray。

\item
概念不会增加运行时的成本。
\end{itemize}

最初,概念是C++11的关键特性,但在2009年7月法兰克福的标准化会议上删除了。这里引用Bjarne Stroustrup对当时概念的看法:\href{https://isocpp.org/blog/2013/02/concepts-lite-constraining-templates-with-predicates-andrew-sutton-bjarne-s}{“C++0x的概念变成一个复杂的怪物”}。几年后,第二次尝试也没有成功:精简版概念从C++17标准中删除了。最终,成了C++20的一部分。

\subsubsubsection{4.1.4\hspace{0.2cm}概念的使用}

可以用四种方式来概念。

\hspace*{\fill} \\ %插入空行
\noindent
\textbf{4.1.4.1\hspace{0.2cm}四种方式}

我在conceptsIntegralVariations.cpp中演示了这四种方式,使用了预定义的概念std::integral。

\hspace*{\fill} \\ %插入空行
\noindent
\textbf{使用std::integral概念的四种方式}
\begin{lstlisting}[style=styleCXX]
// conceptsIntegralVariations.cpp

#include <concepts>
#include <iostream>

template<typename T>
requires std::integral<T>
auto gcd(T a, T b) {
	if( b == 0 ) return a;
	else return gcd(b, a % b);
}

template<typename T>
auto gcd1(T a, T b) requires std::integral<T> {
	if( b == 0 ) return a;
	else return gcd1(b, a % b);
}

template<std::integral T>
auto gcd2(T a, T b) {
	if( b == 0 ) return a;
	else return gcd2(b, a % b);
}

auto gcd3(std::integral auto a, std::integral auto b) {
	if( b == 0 ) return a;
	else return gcd3(b, a % b);
}

int main(){

	std::cout << '\n';

	std::cout << "gcd(100, 10)= " << gcd(100, 10) << '\n';
	std::cout << "gcd1(100, 10)= " << gcd1(100, 10) << '\n';
	std::cout << "gcd2(100, 10)= " << gcd2(100, 10) << '\n';
	std::cout << "gcd3(100, 10)= " << gcd3(100, 10) << '\n';

	std::cout << '\n';

}
\end{lstlisting}

因为第3行使用了<concepts>,所以可以在这里使用std::integral概念。若T是类型\href{https://en.cppreference.com/w/cpp/types/is_integral}{integral},则匹配了这个概念。gcd函数表示基于\href{https://en.wikipedia.org/wiki/Euclid}{Euclidean}的最大公约数算法。

下面是使用概念的四种方式:

\begin{itemize}
\item
requires子句(第7行)

\item
尾部requires子句(第14行)

\item
约束模板参数(第19行)

\item
简化的函数模板(第25行)
\end{itemize}

简单起见,每个函数模板只返回auto。函数模板gcd、gcd1、gcd2和函数gcd3之间存在语义差异。对于gcd、gcd1或gcd2,参数a和b必须具有相同的类型,而函数gcd3不同。参数a和b可以有不同的类型,但必须都满足std::integral概念。

\begin{tcblisting}{commandshell={}}
gcd(100, 10) = 10
gcd1(100, 10) = 10
gcd2(100, 10) = 10
gcd3(100, 10) = 10
\end{tcblisting}

gcd和gcd1使用的函数requires子句,其实很强大。

\hspace*{\fill} \\ %插入空行
\noindent
\textbf{4.1.4.2\hspace{0.2cm}requires子句}

conceptsIntegralVariations.cpp演示了如何使用概念来定义函数或函数模板。为了完整起见,我想补充一点:可以使用概念指定函数或函数模板的返回类型。

关键字requires引入了一个requires子句,指定了模板参数(gcd)或函数声明(gcd1)上的约束。requires后面必须跟一个编译时谓词,例如概念(gcd)、概念的连接/析取,或是requires表达式。

编译时谓词也可以是表达式:

\hspace*{\fill} \\ %插入空行
\noindent
\textbf{requires子句中使用编译时谓词}
\begin{lstlisting}[style=styleCXX]
// requiresClause.cpp

#include <iostream>

template <unsigned int i>
requires (i <= 20)
int sum(int j) {
	return i + j;
}


int main() {

	std::cout << '\n';

	std::cout << "sum<20>(2000): " << sum<20>(2000) << '\n',
	// std::cout << "sum<23>(2000): " << sum<23>(2000) << '\n', // ERROR

	std::cout << '\n';

}
\end{lstlisting}

第6行中使用的编译时谓词:需要用于非类型的i,而不非普通的类型。

\begin{tcblisting}{commandshell={}}
sum<20>(2000): 2020
\end{tcblisting}

当打开第17行的注释时,clang编译器报告以下错误:

\begin{center}
\includegraphics[width=1.0\textwidth]{content/3/chapter4/images/1-2.png}\\
requires子句中使用编译时谓词失败
\end{center}

\begin{tcolorbox}[breakable,enhanced jigsaw,colback=blue!5!white,colframe=blue!75!black,title={避免在requires子句中使用编译时谓词}]
当使用概念约束模板参数或函数模板时,应该使用命名概念或组合它们。概念属于语义范畴,而不是像i <= 20这样的语法约束。并且,为概念命名可以使其重用。
\end{tcolorbox}

\hspace*{\fill} \\ %插入空行
\noindent
\textbf{4.1.4.3\hspace{0.2cm}函数的返回类型为概念}

下面是使用概念作为返回类型的函数模板gcd和函数gcd1的定义。

\hspace*{\fill} \\ %插入空行
\noindent
\textbf{概念作为返回类型}
\begin{lstlisting}[style=styleCXX]
template<typename T>
requires std::integral<T>
std::integral auto gcd(T a, T b) {
	if( b == 0 ) return a;
	else return gcd(b, a % b);
}

std::integral auto gcd1(std::integral auto a, std::integral auto b) {
	if( b == 0 )return a;
	else return gcd1(b, a % b);
}
\end{lstlisting}

\hspace*{\fill} \\ %插入空行
\noindent
\textbf{4.1.4.4\hspace{0.2cm}概念的用例}

首先,概念是编译时谓词。编译时谓词是在编译时执行,并返回布尔值的函数。深入研究概念的各种用例前,我想先揭开概念的神秘面纱,并将它们简单地表示为在编译时返回布尔值的函数。

\hspace*{\fill} \\ %插入空行
\noindent
\textbf{4.1.4.4.1\hspace{0.2cm}编译时谓词}

概念可用于在运行时或编译时执行的控制结构中。

\hspace*{\fill} \\ %插入空行
\noindent
\textbf{概念作为编译时谓词}
\begin{lstlisting}[style=styleCXX]
// compileTimePredicate.cpp

#include <compare>
#include <iostream>
#include <string>
#include <vector>

struct Test{};

int main() {

	std::cout << '\n';

	std::cout << std::boolalpha;

	std::cout << "std::three_way_comparable<int>: "
	          << std::three_way_comparable<int> << "\n";

	std::cout << "std::three_way_comparable<double>: ";
	if (std::three_way_comparable<double>) std::cout << "True";
	else std::cout << "False";

	std::cout << "\n\n";

	static_assert(std::three_way_comparable<std::string>);

	std::cout << "std::three_way_comparable<Test>: ";
	if constexpr(std::three_way_comparable<Test>) std::cout << "True";
	else std::cout << "False";

	std::cout << '\n';

	std::cout << "std::three_way_comparable<std::vector<int>>: ";
	if constexpr(std::three_way_comparable<std::vector<int>>) std::cout << "True";
	else std::cout << "False";

	std::cout << '\n';

}
\end{lstlisting}

例子中,我使用了std::three\_way\_comparable<T>这个概念,在比较时检查T是否支持六个比较运算符。作为编译时谓词std::thre\_way\_comparable可以在运行时(第17行和第20行)或编译时使用。static\_assert(第25行)和\href{https://en.cppreference.com/w/cpp/language/if}{constepr if}(第28和34行)可在编译时进行计算。

\begin{tcblisting}{commandshell={}}
std::three_way_coparable<int>: True
std::three_way_coparable<double>: True

std::three_way_coparable<Test>: False
std::three_way_coparable<std::vector<int>>: True
\end{tcblisting}

对编译时谓词的概念进行简短的介绍之后,继续本节的概念的各种用例。这些概念的应用不是很详细,这里主要使用的是预定义概念,我将后续的章节中讨论这些概念。

\hspace*{\fill} \\ %插入空行
\noindent
\textbf{4.1.4.4.2\hspace{0.2cm}类模板}

类模板MyVector要求模板形参T是常规类型,所以T的行为类似于int。\href{https://en.cppreference.com/w/cpp/concepts/regular}{std::regular}的定义在之前有提到。

\hspace*{\fill} \\ %插入空行
\noindent
\textbf{类定义中使用概念}
\begin{lstlisting}[style=styleCXX]
// conceptsClassTemplate.cpp

#include <concepts>
#include <iostream>

template <std::regular T>
class MyVector{};

int main() {

	MyVector<int> myVec1;
	MyVector<int&> myVec2; // ERROR because a reference is not regular

}
\end{lstlisting}

因为引用非常规类型,所以第12行会出现编译错误。下面是GCC编译器错误消息的重要部分:

\begin{center}
\includegraphics[width=1.0\textwidth]{content/3/chapter4/images/1-3.png}\\
引用非常规类型
\end{center}

\hspace*{\fill} \\ %插入空行
\noindent
\textbf{4.1.4.4.3\hspace{0.2cm}泛型成员函数}

这个例子中,向MyVector类添加了一个通用的push\_back成员函数。push\_back要求参数可复制。

\hspace*{\fill} \\ %插入空行
\noindent
\textbf{泛型成员函数中使用概念}
\begin{lstlisting}[style=styleCXX]
// conceptMemberFunction.cpp

#include <concepts>
#include <iostream>

struct NotCopyable {
	NotCopyable() = default;
	NotCopyable(const NotCopyable&) = delete;
};

template <typename T>
struct MyVector{
	void push_back(const T&) requires std::copyable<T> {}
};

int main() {

	MyVector<int> myVec1;
	myVec1.push_back(2020);

	MyVector<NotCopyable> myVec2;
	myVec2.push_back(NotCopyable()); // ERROR because not copyable

}
\end{lstlisting}

编译在第22行失败。因为复制构造函数声明为已删除,所以NotCopyable的实例不可复制。

\hspace*{\fill} \\ %插入空行
\noindent
\textbf{4.1.4.4.4\hspace{0.2cm}可变模板}

可变模板中可以使用概念。

\begin{lstlisting}[style=styleCXX]
// allAnyNone.cpp

#include <concepts>
#include <iostream>

template<std::integral... Args>
bool all(Args... args) { return (... && args); }

template<std::integral... Args>
bool any(Args... args) { return (... || args); }

template<std::integral... Args>
bool none(Args... args) { return not(... || args); }

int main(){

	std::cout << std::boolalpha << '\n';

	std::cout << "all(5, true, false): " << all(5, true, false) << '\n';

	std::cout << "any(5, true, false): " << any(5, true, false) << '\n';

	std::cout << "none(5, true, false): " << none(5, true, false) << '\n';

}
\end{lstlisting}

上述函数模板的定义基于折叠表达式。C++11支持可变参数模板,可以接受任意数量的模板参数,任意数量的模板参数由一个所谓的参数包保存。C++17中可以使用二进制操作符直接简化参数包,这种简化称为\href{https://www.modernescpp.com/index.php/fold-expressions}{折叠表达式}。

在本例中,逻辑和\&\&(第7行)、逻辑或||(第10行)以及逻辑或的否定(第13行)作为二元运算符。此外,all、any和none要求类型参数必须支持std::integral概念。

\begin{tcblisting}{commandshell={}}
all(5, true, false): false
any(5, true, false): true
none(5, ture, false): false
\end{tcblisting}

\hspace*{\fill} \\ %插入空行
\noindent
\textbf{4.1.4.4.5\hspace{0.2cm}重载}

\href{https://en.cppreference.com/w/cpp/iterator/advance}{std::advance}是标准模板库的算法,对给定的迭代器iter增加n个元素,根据给定迭代器的功能,可以使用不同的策略。例如,std::forward\_list支持只能朝一个方向前进的迭代器,而std::list支持双向迭代器,std::vector支持随机访问迭代器。因此,对于std::forward\_list或std::list提供的迭代器,对std::advance(iter, n)的调用必须加n倍(参见std::list的结构)耗时。这个时间复杂度不适用于std::vector提供的std::random\_access\_iterator,数字n可以直接加到迭代器中。因此,线性时间复杂度O(n)变成了O(1)。为了区分迭代器类型,可以使用概念。conceptsOverloadingFunctionTemplates.cpp展示了这样使用概念的方式。

\begin{lstlisting}[style=styleCXX]
// conceptsOverloadingFunctionTemplates.cpp

#include <concepts>
#include <iostream>
#include <forward_list>
#include <list>
#include <vector>

template<std::forward_iterator I>
void advance(I& iter, int n){
	std::cout << "forward_iterator" << '\n';
}

template<std::bidirectional_iterator I>
void advance(I& iter, int n){
	std::cout << "bidirectional_iterator" << '\n';
}

template<std::random_access_iterator I>
void advance(I& iter, int n){
	std::cout << "random_access_iterator" << '\n';
}

int main() {

	std::cout << '\n';

	std::forward_list forwList{1, 2, 3};
	std::forward_list<int>::iterator itFor = forwList.begin();
	advance(itFor, 2);

	std::list li{1, 2, 3};
	std::list<int>::iterator itBi = li.begin();
	advance(itBi, 2);

	std::vector vec{1, 2, 3};
	std::vector<int>::iterator itRa = vec.begin();
	advance(itRa, 2);

	std::cout << '\n';
}
\end{lstlisting}

函数advance的三种重载位于std::forward\_iterator(第9行)、std::bidirectional\_iterator(第14行)和std::random\_access\_iterator(第19行)这三个概念上,编译器会选择最合适的重载。所以,对于std::forward\_list(第28行),对应的是基于std::forward\_iterator概念的重载;对于std::list(第32行),对应的是基于std::bidirectional\_iterator概念的重载;对于std::vector(第36行),对应的是基于std::random\_access\_iterator概念的重载。

\begin{tcblisting}{commandshell={}}
forward_iterator
bidirectional_iterator
random_access_iterator
\end{tcblisting}

std::random\_access\_iterator可以当作std::bidirectional\_iterator看待,std::bidirectional\_iterator可以当作std::forward\_iterator看待。

\hspace*{\fill} \\ %插入空行
\noindent
\textbf{4.1.4.4.6\hspace{0.2cm}特化的模板}

还可以使用概念特化的模板。

\begin{lstlisting}[style=styleCXX]
// conceptsSpecialization.cpp

#include <concepts>
#include <iostream>

template <typename T>
struct Vector {
	Vector() {
		std::cout << "Vector<T>" << '\n';
	}
};

template <std::regular Reg>
struct Vector<Reg> {
	Vector() {
		std::cout << "Vector<std::regular>" << '\n';
	}
};

int main() {

	std::cout << '\n';

	Vector<int> myVec1;
	Vector<int&> myVec2;

	std::cout << '\n';

}
\end{lstlisting}

实例化类模板时,编译器会选择最特化的一个。对于Vector<int> myVec(第24行),会选择std::regular(第13行)的偏特化模板。引用Vector<int\&> myVec2(第25行)不是std::regular类型,因此选择主模板(第6行)。

\begin{tcblisting}{commandshell={}}
Vector<std::regular>
Vector<T>
\end{tcblisting}

\hspace*{\fill} \\ %插入空行
\noindent
\textbf{4.1.4.4.7\hspace{0.2cm}使用多个概念}

目前,这些概念的使用都很简单,也可以同时使用多个概念。

\begin{lstlisting}[style=styleCXX]
template<typename Iter, typename Val>
	requires std::input_iterator<Iter>
		  && std::equality_comparable<Value_type<Iter>, Val>
Iter find(Iter b, Iter e, Val v)
\end{lstlisting}

find需要Iter迭代器与Val进行比较

\begin{itemize}
\item
迭代器必须是输入迭代器;

\item
迭代器的值类型必须与Val相同。
\end{itemize}

对迭代器的限制也可以表示为受约束的模板形参。

\begin{lstlisting}[style=styleCXX]
template<std::input_iterator Iter, typename Val>
	requires std::equality_comparable<Value_type<Iter>, Val>
Iter find(Iter b, Iter e, Val v)
\end{lstlisting}

\subsubsubsection{4.1.5\hspace{0.2cm}约束和非约束占位符}

首先,让我告诉你C++14中的一个不对称。

\hspace*{\fill} \\ %插入空行
\noindent
\textbf{4.1.5.1\hspace{0.2cm}C++14中的不对称}

我经常在课堂上进行讨论。C++14中,我们有泛型Lambda,可以使用auto。

\hspace*{\fill} \\ %插入空行
\noindent
\textbf{泛型Lambda和函数模板的比较}
\begin{lstlisting}[style=styleCXX]
// genericLambdaTemplate.cpp

#include <iostream>
#include <string>

auto addLambda = [](auto fir, auto sec){ return fir + sec; };

template <typename T, typename T2>
auto addTemplate(T fir, T2 sec){ return fir + sec; }

int main(){

	std::cout << std::boolalpha << '\n';

	std::cout << addLambda(1, 5) << " " << addTemplate(1, 5) << '\n';
	std::cout << addLambda(true, 5) << " " << addTemplate(true, 5) << '\n';
	std::cout << addLambda(1, 5.5) << " " << addTemplate(1, 5.5) << '\n';

	const std::string fir{"ge"};
	const std::string sec{"neric"};
	std::cout << addLambda(fir, sec) << " " << addTemplate(fir, sec) << '\n';

	std::cout << '\n';

}
\end{lstlisting}

泛型Lambda(第6行)和函数模板(第8行)产生相同的结果。

\begin{center}
\includegraphics[width=0.6\textwidth]{content/3/chapter4/images/8.png}\\
\end{center}

泛型Lambda引入了一种定义函数模板的新方法。在课堂上,经常有人问:我们可以在函数中使用auto获取函数模板吗?C++14不行,但C++20可以。

C++20中,可以在函数声明中使用无约束占位符(auto)或有约束占位符(概念)来自动获取函数模板。规则非常简单,在每个可以使用无约束占位符auto的地方,都可以使用一个概念。我将在简写函数模板一节中进行详细说明。

\hspace*{\fill} \\ %插入空行
\noindent
\textbf{4.1.5.2\hspace{0.2cm}占位符}

\hspace*{\fill} \\ %插入空行
\noindent
\textbf{使用约束占位符}
\begin{lstlisting}[style=styleCXX]
// placeholders.cpp

#include <concepts>
#include <iostream>
#include <vector>

std::integral auto getIntegral(int val){
	return val;
}

int main(){

	std::cout << std::boolalpha << '\n';

	std::vector<int> vec{1, 2, 3, 4, 5};
	for (std::integral auto i: vec) std::cout << i << " ";
	std::cout << '\n';

	std::integral auto b = true;
	std::cout << b << '\n';

	std::integral auto integ = getIntegral(10);
	std::cout << integ << '\n';

	auto integ1 = getIntegral(10);
	std::cout << integ1 << '\n';

	std::cout << '\n';

}
\end{lstlisting}

std::integral概念可以用作返回类型(第7行)、基于范围的for循环(第16行),或者用作变量b(第19行)或变量integ(第22行)的类型。为了查看auto和概念之间的对称性,第25行单独使用auto,而不是在第22行使用的std::integral auto。因此,integ1可以接受任何类型的值。

\begin{tcblisting}{commandshell={}}
1 2 3 4 5
true
10
10
\end{tcblisting}

\subsubsubsection{4.1.6\hspace{0.2cm}简写函数模板}

C++20中,可以在函数声明中使用不受约束的占位符(auto)或受约束的占位符(concept),这个函数声明会自动成为一个函数模板。

\begin{lstlisting}[style=styleCXX]
// abbreviatedFunctionTemplates.cpp

#include <concepts>
#include <iostream>

template<typename T>
requires std::integral<T>
T gcd(T a, T b) {
	if( b == 0 ) return a;
	else return gcd(b, a % b);
}

template<typename T>
T gcd1(T a, T b) requires std::integral<T> {
	if( b == 0 ) return a;
	else return gcd1(b, a % b);
}

template<std::integral T>
T gcd2(T a, T b) {
	if( b == 0 ) return a;
	else return gcd2(b, a % b);
}

std::integral auto gcd3(std::integral auto a, std::integral auto b) {
	if( b == 0 ) return a;
	else return gcd3(b, a % b);
}

auto gcd4(auto a, auto b){
	if( b == 0 ) return a;
	return gcd4(b, a % b);
}

int main() {

	std::cout << '\n';

	std::cout << "gcd(100, 10)= " << gcd(100, 10) << '\n';
	std::cout << "gcd1(100, 10)= " << gcd1(100, 10) << '\n';
	std::cout << "gcd2(100, 10)= " << gcd2(100, 10) << '\n';
	std::cout << "gcd3(100, 10)= " << gcd3(100, 10) << '\n';
	std::cout << "gcd4(100, 10)= " << gcd4(100, 10) << '\n';

	std::cout << '\n';

}
\end{lstlisting}

函数模板gcd(第6行)、gcd1(第13行)和gcd2(第19行)在使用概念的四种方式时已经介绍过了。gdc使用requires子句,gcd1使用尾部requires子句,gcd2使用约束模板参数。现在来看点新东西,函数模板gcd3有std::integral的概念作为类型参数,因此是一个具有受限类型参数的函数模板。相比之下,gcd4相当于对其类型参数,对函数模板没有限制。gcd3和gcd4中用于创建函数模板的语法,称为缩写函数模板。

\begin{tcblisting}{commandshell={}}
gcd(100, 10)= 10
gcd1(100, 10)= 10
gcd2(100, 10)= 10
gcd3(100, 10)= 10
gcd4(100, 10)= 10
\end{tcblisting}

通过下面的例子来强调这种对称性。

使用auto作为类型参数,函数add变成了一个函数模板,与同名的函数模板add等价。

\begin{lstlisting}[style=styleCXX]
template<typename T, typename T2>
auto add(T fir, T2 sec) {
	return fir + sec;
}

auto add(auto fir, auto sec) {
	return fir + sec;
}
\end{lstlisting}

相应地,由于std::integral概念的使用,sub函数等价于函数模板sub函数。

\begin{lstlisting}[style=styleCXX]
template<std::integral T, std::integral T2>
std::integral auto sub(T fir, T2 sec) {
	return fir - sec;
}

std::integral auto sub(std::integral auto fir, std::integral auto sec) {
	return fir - sec;
}
\end{lstlisting}

函数和函数模板可以是任意类型,这两种类型可以是不同的,但必须是整型。例如,使用sub(100, 10)和sub(100, true)都可以。

缩写函数模板语法中,仍然缺少一个特性:可以重载auto或概念。

\hspace*{\fill} \\ %插入空行
\noindent
\textbf{4.1.6.1\hspace{0.2cm}重载}

以下函数会在auto、std::integral概念和long类型上进行重载。

\begin{lstlisting}[style=styleCXX]
// conceptsOverloading.cpp

#include <concepts>
#include <iostream>

void overload(auto t){
	std::cout << "auto : " << t << '\n';
}

void overload(std::integral auto t){
	std::cout << "Integral : " << t << '\n';
}

void overload(long t){
	std::cout << "long : " << t << '\n';
}

int main(){

	std::cout << '\n';

	overload(3.14);
	overload(2010);
	overload(2020L);

	std::cout << '\n';

}
\end{lstlisting}

编译器选择auto(第6行)上的重载使用double,std::integral(第10行)上的重载使用int,long(第14行)上的重载使用long。

\begin{tcblisting}{commandshell={}}
auto : 3.14
Integral : 2010
long : 2020
\end{tcblisting}

\begin{tcolorbox}[breakable,enhanced jigsaw,colback=blue!5!white,colframe=blue!75!black,title={遗漏的特性:模板}]
也许在这一章的概念中遗漏了一个特性:模板。将模板引入是概念技术规范的一部分,\href{https://www.iso.org/standard/64031.html}{TS ISO/IEC TS 19217:2015}是概念的实验性实现。\href{https://en.wikipedia.org/wiki/GNU_Compiler_Collection}{GCC 6}完全实现了概念TS,除了语法上与C++20中的概念不同之外,概念TS支持一种简洁的定义模板的方式。

\hspace*{\fill} \\ %插入空行
下面的例子中,假设Integral是一个概念。

\hspace*{\fill} \\ %插入空行
\noindent
\textbf{概念TS中的模板介绍}
\begin{lstlisting}[style=styleCXX]
Integral{T}
Integral gcd(T a, T b){
	if( b == 0 ){ return a; }
	else{
		return gcd(b, a % b);
	}
}

Integral{T}
class ConstrainedClass{};
\end{lstlisting}

上面的这个小代码片段,以两种方式引入了模板。首先,定义一个带有约束模板参数的函数模板;其次,定义带有约束模板参数的类模板。引入模板有一个限制,只能将其用于有约束的模板参数(concept),而不能用于无约束的模板参数(auto)。这种不对称性可以通过定义一个总是返回true的概念轻松搞定:

\hspace*{\fill} \\ %插入空行
\noindent
\textbf{Generic概念的实现}
\begin{lstlisting}[style=styleCXX]
template<typename T>
concept bool Generic(){
	return true;
}
\end{lstlisting}

不要着急,我在示例中使用概念TS语法来定义泛型概念。C++20的语法会简洁一些。在定义概念一节中可以了解更多的C++20语法细节。
\end{tcolorbox}

\subsubsubsection{4.1.7\hspace{0.2cm}预定义的概念}

“不要白费力气”的黄金法则同样适用于概念。\href{https://isocpp.github. io/CppCoreGuidelines/CppCoreGuidelines}{C++核心指南}对这条规则的定义非常清楚:\textit{T.11:只要可能,就使用标准概念}。因此,我想给出一个重要的预定义概念的概述,这里会有意地忽略特殊或辅助类型的概念。

所有预定义的概念在最新的C++20工作草案\href{https://isocpp.org/files/papers/N4860.pdf}{N4860}中都有详细描述,找到它们是一个相当大的挑战!大部分概念都在第18章(概念库)和第24章(范围库)中。另外,第17章(语言支持库)、第20章(通用实用程序库)、第23章(迭代器库)和第26章(数字库)中有一些概念。C++20草案N4860还提供了所有库概念的索引,并展示了如何实现这些概念。

\hspace*{\fill} \\ %插入空行
\noindent
\textbf{4.1.7.1\hspace{0.2cm}语言支持库}

本节讨论一个有趣的概念——three\_way\_comparable,支持三向比较运算符,在头文件<compare>中定义。

更正式地说,设a和b是类型为t的值。只有在以下情况下,这些值才支持three\_way\_comparable:

\begin{tcblisting}{commandshell={}}
• (a <=> b == 0) == bool(a == b) is true
• (a <=> b != 0) == bool(a != b) is true
• ((a <=> b) <=> 0) and (0 <=> (b <=> a)) are equal
• (a <=> b < 0) == bool(a < b) is true
• (a <=> b > 0) == bool(a > b) is true
• (a <=> b <= 0) == bool(a <= b) is true
• (a <=> b >= 0) == bool(a >= b) is true
\end{tcblisting}

\hspace*{\fill} \\ %插入空行
\noindent
\textbf{4.1.7.2\hspace{0.2cm}概念库}

最常用的概念可以在概念库中找到,在<concepts>头文件中定义。

\hspace*{\fill} \\ %插入空行
\noindent
\textbf{4.1.7.2.1\hspace{0.2cm}语言相关的概念}

本节有大约15个概念,这些概念表示类型、类型分类和基本类型属性之间的关系。其实现通常直接基于\href{https://en.cppreference.com/w/cpp/header/type_traits}{类型特性库}中的相应函数。若有必要,我会提供额外的解释。

\begin{itemize}
\item
default\_initializable: default\_initializable<T>保证T可以默认构造

\item
same\_as

\item
derived\_from

\item
convertible\_to: convertible\_to<T, U>保证T可以隐式或显式转换为U

\item
common\_reference\_with: common\_reference\_with<T, U>必须定义良好,T和U必须可以转换为引用类型C,其中C与common\_reference\_t<T, U>相同

\item
common\_with: 类似于common\_reference\_with,但是common类型C与common\_type\_t<T, U>相同,并且可能不是引用类型

\item
assignable\_from

\item
swappable
\end{itemize}

\hspace*{\fill} \\ %插入空行
\noindent
\textbf{4.1.7.2.2\hspace{0.2cm}算术的概念}

\begin{itemize}
\item
integral

\item
signed\_integral

\item
unsigned\_integral

\item
floating\_point
\end{itemize}

该标准对算术概念的定义很简单:

\begin{lstlisting}[style=styleCXX]
template<class T>
concept integral = is_integral_v<T>;

template<class T>
concept signed_integral = integral<T> && is_signed_v<T>;

template<class T>
concept unsigned_integral = integral<T> && !signed_integral<T>;

template<class T>
concept floating_point = is_floating_point_v<T>;
\end{lstlisting}

\hspace*{\fill} \\ %插入空行
\noindent
\textbf{4.1.7.2.3\hspace{0.2cm}生命周期的概念}

\begin{itemize}
\item
destructible

\item
constructible\_from

\item
default\_constructible

\item
move\_constructible

\item
copy\_constructible
\end{itemize}

\hspace*{\fill} \\ %插入空行
\noindent
\textbf{4.1.7.2.4\hspace{0.2cm}比较的概念}

\begin{itemize}
\item
equality\_comparable

\item
totally\_ordered
\end{itemize}

学习数学时会了解:对于类型T的值a、b和c,若以下描述成立,则T类型为totally\_ordered(全序关系)

\begin{itemize}
\item
bool(a < b)、bool(a > b)或bool(a == b)中的一个为真

\item
bool(a < b)和bool(b < c),则bool(a < c)

\item
bool(a > b) == bool(b < a)

\item
bool(a <= b) == !bool(b < a)

\item
bool(a >= b) == !bool(a < b)
\end{itemize}

\hspace*{\fill} \\ %插入空行
\noindent
\textbf{4.1.7.2.5\hspace{0.2cm}对象的概念}

\begin{itemize}
\item
movable

\item
copyable

\item
semiregular

\item
regular
\end{itemize}

以下是这四个概念的定义:

\begin{lstlisting}[style=styleCXX]
template<class T>
concept movable = is_object_v<T> && move_constructible<T> &&
					assignable_from<T&, T> && swappable<T>;

template<class T>
concept copyable = copy_constructible<T> && movable<T> &&
					assignable_from<T&, T&> &&
					assignable_from<T&, const T&> && assignable_from<T&, const T>;

template<class T>
concept semiregular = copyable<T> && default_initializable<T>;

template<class T>
concept regular = semiregular<T> && equality_comparable<T>;
\end{lstlisting}

我必须补充几点。可移动的概念要求T的is\_object\_v<T>条件成立。根据类型特性is\_object<T>的定义,T可以是标量、数组、联合体或者类。

定义概念部分还实现了半常规和常规的概念。不太正式地说,半常规类型的行为类似于int类型;而常规类型的行为也类似于int类型,并且可以使用==操作符进行比较。

\hspace*{\fill} \\ %插入空行
\noindent
\textbf{4.1.7.2.6\hspace{0.2cm}可调用的概念}

\begin{itemize}
\item
invocable

\item
regular\_invocable:类型建模为invocable且保持相等,并且不修改函数参数;保持相等,说明在给定相同的输入时会产生相同的输出

\item
predicate:若类型对invocable建模并返回布尔值,则可对谓词建模
\end{itemize}

\hspace*{\fill} \\ %插入空行
\noindent
\textbf{4.1.7.3\hspace{0.2cm}通用工具库}

本章在标准中只有特殊的内存概念,因此在这里不提及它们。

\hspace*{\fill} \\ %插入空行
\noindent
\textbf{4.1.7.4\hspace{0.2cm}迭代器库}

迭代器库有许多重要的概念,在<iterator>头文件中定义。下面是迭代器的类别:

\begin{itemize}
\item
input\_iterator

\item
output\_iterator

\item
forward\_iterator

\item
bidirectional\_iterator

\item
random\_access\_iterator

\item
contiguous\_iterator
\end{itemize}

这六类迭代器对应于各自的迭代器概念。下表提供了两条有趣的信息。对于三个最突出的迭代器类别,该表显示了它们的属性和相关的标准库容器。

\begin{table}[H]
\centering
\begin{tabular}{lll}
\textbf{迭代器类别}   & \textbf{属性} & \textbf{相应容器}                                                                                     \\ \hline
std::forward\_iterator &
\begin{tabular}[c]{@{}l@{}}++It,It++,*It\\ It==It2, It!=It2\end{tabular} &
\begin{tabular}[c]{@{}l@{}}std::unordered\_set\\ std::unordered\_map\\ std::unordered\_multiset\\ std::unordered\_multimap\\ std::forward\_list\end{tabular} \\ \hline
std::bidirectional\_iterator & -{}-It,It-{}-           & \begin{tabular}[c]{@{}l@{}}std::set\\ std::map\\ std::multiset\\ std::multimap\\ std::list\end{tabular} \\ \hline
std::random\_access\_iterator &
\begin{tabular}[c]{@{}l@{}}It{[}i{]}\\ It += n, It -= n\\ It + n, It - n,\\ n + It,\\ It - It2,\\ It \textless It2, It \textless{}= It2\\ It \textless It2, It \textgreater{}= It2\end{tabular} &
\begin{tabular}[c]{@{}l@{}}std::deque\end{tabular} \\ \hline
std::contiguous\_iterator &
\begin{tabular}[c]{@{}l@{}}It{[}i{]}\\ It += n, It -= n\\ It + n, It - n,\\ n + It,\\ It - It2,\\ It \textless It2, It \textless{}= It2\\ It \textless It2, It \textgreater{}= It2\end{tabular} &
\begin{tabular}[c]{@{}l@{}}std::array\\ std::vector\\ std::string\end{tabular}
\end{tabular}
\end{table}

以下关系成立:

\begin{itemize}
\item
随机访问迭代器是双向迭代器,双向迭代器是前向迭代器。

\item
连续迭代器是一种随机访问迭代器,要求容器的元素连续存储在内存中。
\end{itemize}

所以std::array,std::vector和std::string支持连续迭代器,但不包括std::deque。

\hspace*{\fill} \\ %插入空行
\noindent
\textbf{4.1.7.4.1\hspace{0.2cm}算法的概念}

\begin{itemize}
\item
permutable:元素可直接重排序

\item
mergeable:可以将已排序的序列合并到输出序列中

\item
sortable:可以将一个序列排列成有序序列
\end{itemize}

\hspace*{\fill} \\ %插入空行
\noindent
\textbf{4.1.7.5\hspace{0.2cm}范围库}

范围库包含对范围和视图特性至关重要的概念,类似于迭代器库中的概念,定义在<ranges>头文件中。

\hspace*{\fill} \\ %插入空行
\noindent
\textbf{4.1.7.5.1\hspace{0.2cm}范围}

\begin{itemize}
\item
range:范围指定可以遍历的一组项,提供了一个开始迭代器和一个结束哨兵。当然,STL容器也有范围。
\begin{lstlisting}[style=styleCXX]
template<typename T>
concept range = requires(T& t) {
	ranges::begin(t);
	ranges::end (t);
};
\end{lstlisting}
\end{itemize}

对于std::ranges::range还可以进一步的细化。

\begin{itemize}
\item
input\_range:指定一个范围,其迭代器类型满足input\_iterator(例如,可以从开始迭代到结束至少一次)

\item
output\_range:指定迭代器类型满足output\_iterator的范围

\item
forward\_range:指定一个范围,其迭代器类型满足forward\_iterator(可以从开始到结束迭代多次)

\item
bidirectional\_range:指定迭代器类型满足bidirectional\_iterator的范围(可以向前和向后迭代不止一次)

\item
random\_access\_range:指定迭代器类型满足random\_access\_iterator的范围(可以在常量时间内使用索引操作符[]跳访问任意元素)

\item
contiguous\_range:指定一个范围,其迭代器类型满足contiguous\_iterator(元素连续存储在内存中)
\end{itemize}

标准模板库的每个容器都支持特定的范围,支持的范围指定了其迭代器的功能。

\begin{table}[H]
\centering
\begin{tabular}{lll}
\textbf{迭代器类型}        & \textbf{属性} & \textbf{相应容器}                                                                                     \\ \hline
std::ranges::input\_range &
\begin{tabular}[c]{@{}l@{}}++It,It++,*It\\ It==It2, It!=It2\end{tabular} &
\begin{tabular}[c]{@{}l@{}}std::unordered\_set\\ std::unordered\_map\\ std::unordered\_multiset\\ std::unordered\_multimap\\ std::forward\_list\end{tabular} \\ \hline
std::ranges::bidirectional\_range & -{}-It,It-{}-           & \begin{tabular}[c]{@{}l@{}}std::set\\ std::map\\ std::multiset\\ std::multimap\\ std::list\end{tabular} \\ \hline
std::ranges::random\_access\_range &
\begin{tabular}[c]{@{}l@{}}It{[}i{]}\\ It += n, It -= n\\ It + n, It - n,\\ n + It,\\ It - It2,\\ It \textless It2, It \textless{}= It2\\ It \textless It2, It \textgreater{}= It2\end{tabular} &
\begin{tabular}[c]{@{}l@{}}std::deque\end{tabular} \\ \hline
std::ranges::contiguous\_range &
\begin{tabular}[c]{@{}l@{}}It{[}i{]}\\ It += n, It -= n\\ It + n, It - n,\\ n + It,\\ It - It2,\\ It \textless It2, It \textless{}= It2\\ It \textless It2, It \textgreater{}= It2\end{tabular} &
\begin{tabular}[c]{@{}l@{}}std::array\\ std::vector\\ std::string\end{tabular}
\end{tabular}
\end{table}

若容器支持std::ranges::continuous\_range概念,则支持表中提到的所有概念,如std::ranges::random\_access\_range,std::ranges::bidirectional\_range和std::ranges::input\_range。对于其他范围同理。

有两个特殊范围

\begin{itemize}
\item
sized\_rang:保证可以使用元素的开始和结束的差值在恒定时间内计算元素的数量

\item
borrowed\_range:确保迭代器不绑定到范围的生存期
\end{itemize}

\hspace*{\fill} \\ %插入空行
\noindent
\textbf{4.1.7.5.2\hspace{0.2cm}视图}

std::ranges::view是一个具有恒定时间复制、移动和赋值操作的范围。
以下几行显示了概念视图的定义。

\begin{lstlisting}[style=styleCXX]
template<typename T>
concept view = range<T> &&
		movable<T> &&
		default_initializable<T> &&
		enable_view<T>;
template<class T>
inline constexpr bool enable_view = derived_from<T, view_base>;
\end{lstlisting}

视图通常是应用在一个范围上,并用其执行一些操作。视图不拥有数据,视图用于复制、移动或赋值的时间恒定。它应该是只读的、无状态的和保持相等。这里引用Eric Niebler的range-v3实现(是C++20范围的基础):“视图是范围的组合适配,随着视图的迭代,这种适配的功能会慢慢地发挥其功效。”

因此,STL的容器是范围而不是视图。

\hspace*{\fill} \\ %插入空行
\noindent
\textbf{4.1.7.6\hspace{0.2cm}数值库}

数值库提供了uniform\_random\_bit\_generator的概念,该概念定义在头文件<random>中。类型G的uniform\_random\_bit\_generator g必须返回均分布的无符号整数,类型为G的均匀随机位生成器g必须支持成员函数G::min和G::max。

\subsubsubsection{4.1.8\hspace{0.2cm}定义概念}

当在C++20中没有合适的预定义的概念时,可以自定义概念。在本节中,我将定义几个概念,使用CamelCase语法将它们与预定义的概念区别开来。因此,我的带符号整型的概念命名为SignedIntegral,而C++标准的概念为signed\_integral。

定义概念的语法很简单:

\begin{lstlisting}[style=styleCXX]
template <template-parameter-list>
concept concept-name = constraint-expression;
\end{lstlisting}

概念定义从关键字template开始,并有一个模板参数列表。第二行更有趣,使用关键字concept,后面跟着概念名称和约束表达式。

约束表达式可以是:

\begin{itemize}
\item
概念或编译时谓词的逻辑组合

\begin{itemize}
\item
逻辑组合可以由连接(\&\&)、析取(||)或否定(!)

\item
编译时谓词是在编译时返回布尔值的可调用对象
\end{itemize}

\item
需求表达式
\begin{itemize}
\item
简单的需求

\item
类型的需求

\item
复合的需求

\item
嵌套的需求
\end{itemize}
\end{itemize}

接下来的两节中,将演示定义概念的各种方法。

\hspace*{\fill} \\ %插入空行
\noindent
\textbf{4.1.8.1\hspace{0.2cm}其他概念和编译时谓词的组合}

可以使用连接词(\&\&)和析取词(||)组合概念和编译时谓词。构建逻辑组合时,可以使用感叹号(!)否定。对概念和编译时谓词的这种逻辑组合的评估遵循\href{https://en.wikipedia.org/wiki/Short-circuit_evaluation}{短路求值}. 短路求值意味着逻辑表达式的求值在其总体结果已经确定时自动停止。
因为\href{https://en.cppreference.com/w/cpp/header/type_traits}{类型特性库}有许多编译时谓词,所以具备了使用构建强大概念所需的工具。

\hspace*{\fill} \\ %插入空行
\begin{tcolorbox}[breakable,enhanced jigsaw,colback=red!5!white,colframe=red!75!black,title={不要递归地定义概念或尝试约束它们}]

概念的递归定义无效:

\hspace*{\fill} \\ %插入空行
\noindent
\textbf{递归地定义概念}
\begin{lstlisting}[style=styleCXX]
template<typename T>
concept Recursive = Recursive<T*>;
\end{lstlisting}

GCC编译器在这种情况下抱怨'Recursive'没有在此作用域中声明。

\hspace*{\fill} \\ %插入空行
当尝试约束概念(例如下面的代码片段)时,GCC编译器会明确地提示概念不能约束。

\hspace*{\fill} \\ %插入空行
\noindent
\textbf{约束概念}
\begin{lstlisting}[style=styleCXX]
template<typename T>
concept AlwaysTrue = true;

template<typename T>
requires AlwaysTrue<T>
concept Error = true;
\end{lstlisting}

\end{tcolorbox}

我们先从Integral、SignedIntegral和UnsignedIntegral概念开始。

\begin{lstlisting}[style=styleCXX]
template <typename T>
concept Integral = std::is_integral<T>::value;

template <typename T>
concept SignedIntegral = Integral<T> && std::is_signed<T>::value;

template <typename T>
concept UnsignedIntegral = Integral<T> && !SignedIntegral<T>;
\end{lstlisting}

我使用类型特性函数\href{https://en.cppreference.com/w/cpp/types/is_integral}{std::is\_integral}来定义Integral概念(第2行)。由于函数std::is\_signed,将Integral概念改进为SignedIntegral概念(第4行)。最后,对SignedIntegral概念进行否定,得到了UnsignedIntegral概念(第7行)。

Okay,我们来试试看。

\begin{lstlisting}[style=styleCXX]
// SignedUnsignedIntegrals.cpp

#include <iostream>
#include <type_traits>

template <typename T>
concept Integral = std::is_integral<T>::value;

template <typename T>
concept SignedIntegral = Integral<T> && std::is_signed<T>::value;

template <typename T>
concept UnsignedIntegral = Integral<T> && !SignedIntegral<T>;

void func(SignedIntegral auto integ) {
	std::cout << "SignedIntegral: " << integ << '\n';
}

void func(UnsignedIntegral auto integ) {
	std::cout << "UnsignedIntegral: " << integ << '\n';
}

int main() {

	std::cout << '\n';

	func(-5);
	func(5u);

	std::cout << '\n';

}
\end{lstlisting}

使用简化的函数模板语法重载概念SignedIntegral(第15行)和UnsignedIntegral(第19行)上的函数func。编译器选择预期的重载:

\begin{tcblisting}{commandshell={}}
SignedIntegral: -5
UnsignedIntegral: 5
\end{tcblisting}

出于完整性的原因,会下面的算术概念中使用析取。

\begin{lstlisting}[style=styleCXX]
template <typename T>
concept Arithmetic = std::is_integral<T>::value || std::is_floating_point<T>::value;
\end{lstlisting}

\hspace*{\fill} \\ %插入空行
\noindent
\textbf{4.1.8.2\hspace{0.2cm}需求表达式}

因为需求表达式,现在可以定义功能强大的概念。需求表达式有如下形式:

\begin{lstlisting}[style=styleCXX]
requires (parameter-list(optional)) {requirement-seq}
\end{lstlisting}

\begin{itemize}
\item
参数列表:以逗号分隔的参数列表,例如:在函数声明中

\item
需求序列:由简单需求、类型需求、复合需求或嵌套需求组成
\end{itemize}

当需要编译时谓词时,需求表达式也可以用作独立功能。阅读(需求表达式)中有关此功能的详细信息。

\hspace*{\fill} \\ %插入空行
\noindent
\textbf{4.1.8.2.1\hspace{0.2cm}简单的需求}

下面是概念Addable的简单需求:

\begin{lstlisting}[style=styleCXX]
template<typename T>
concept Addable = requires (T a, T b) {
	a + b;
};
\end{lstlisting}

Addable的概念要求两个相同类型T可以进行加法a + b。

\hspace*{\fill} \\ %插入空行
\noindent
\textbf{4.1.8.2.2\hspace{0.2cm}类型的需求}

类型的需求中,必须使用关键字typename和类型名。

\begin{lstlisting}[style=styleCXX]
template<typename T>
concept TypeRequirement = requires {
	typename T::value_type;
	typename Other<T>;
};
\end{lstlisting}

TypeRequirement概念要求类型T有一个嵌套的成员value\_type,并且类模板Other可以用T实例化。让我们试试这个:

\begin{lstlisting}[style=styleCXX]
#include <iostream>
#include <vector>

template <typename>
struct Other;

template <>
struct Other<std::vector<int>> {};

template<typename T>
concept TypeRequirement = requires {
	typename T::value_type;
	typename Other<T>;
};

int main() {

	TypeRequirement auto myVec= std::vector<int>{1, 2, 3};

}
\end{lstlisting}

表达式TypeRequirement auto myVec = std::vector<int>{1,2,3}(第18行)有效。\href{https://en.cppreference.com/w/cpp/container/vector}{std::vector}有一个内部成员value\_type(第12行),类模板Other可以实例化std::vector<int>(第13行)。

\hspace*{\fill} \\ %插入空行
\noindent
\textbf{4.1.8.2.3\hspace{0.2cm}复合需求}

复合需求有这样的形式

\begin{lstlisting}[style=styleCXX]
{expression} noexcept(optional) return-type-requirement(optional);
\end{lstlisting}

除了简单的需求外,复合需求还可以有\href{https://en.cppreference.com/w/cpp/language/noexcept_spec}{noexcept说明符}和关于其返回类型的需求。

下面的示例中演示了在Equal概念中,使用复合需求。

\begin{lstlisting}[style=styleCXX]
// conceptsDefinitionEqual.cpp

#include <concepts>
#include <iostream>

template<typename T>
concept Equal = requires(T a, T b) {
	{ a == b } -> std::convertible_to<bool>;
	{ a != b } -> std::convertible_to<bool>;
};

bool areEqual(Equal auto a, Equal auto b){
	return a == b;
}

struct WithoutEqual{
	bool operator==(const WithoutEqual& other) = delete;
};

struct WithoutUnequal{
	bool operator!=(const WithoutUnequal& other) = delete;
};

int main() {

	std::cout << std::boolalpha << '\n';
	std::cout << "areEqual(1, 5): " << areEqual(1, 5) << '\n';

	/*

	bool res = areEqual(WithoutEqual(), WithoutEqual());
	bool res2 = areEqual(WithoutUnequal(), WithoutUnequal());

	*/

	std::cout << '\n';

}
\end{lstlisting}

Equal概念(第6行)要求其类型参数T支持相等和不相等操作符。此外,两个操作符都必须返回一个可转换为布尔值的值。当然,int支持Equal概念,但这并不适用于WithoutEqual(第16行)和WithoutUnequal(第20行)类型。因此,当使用WithoutEqual类型时(第31行),在使用GCC编译器时,会得到以下错误消息。

\begin{center}
\includegraphics[width=0.8\textwidth]{content/3/chapter4/images/1-4.png}\\
WithoutEqual不匹配Equal概念
\end{center}

\hspace*{\fill} \\ %插入空行
\noindent
\textbf{4.1.8.2.4\hspace{0.2cm}嵌套需求}

嵌套需求的形式

\begin{lstlisting}[style=styleCXX]
requires constraint-expression;
\end{lstlisting}

嵌套需求用于指定类型参数上的需求。

下面是定义概念UnsignedIntegral的另一种方法(参见概念和谓词的逻辑组合):

\begin{lstlisting}[style=styleCXX]
// nestedRequirements.cpp

#include <type_traits>

template <typename T>
concept Integral = std::is_integral<T>::value;

template <typename T>
concept SignedIntegral = Integral<T> && std::is_signed<T>::value;

// template <typename T>
// concept UnsignedIntegral = Integral<T> && !SignedIntegral<T>;

template <typename T>
concept UnsignedIntegral = Integral<T> &&
requires(T) {
	requires !SignedIntegral<T>;
};

int main() {

	UnsignedIntegral auto n = 5u; // works
	// UnsignedIntegral auto m = 5; // compile time error, 5 is a signed literal

}
\end{lstlisting}

第14行使用SignedIntegral概念,嵌套需求来细化Integral概念。老实说,第11行中注释掉的概念UnsignedIntegral阅读起来更方便。

下一节中的有序概念演示了嵌套需求的使用。

\subsubsubsection{4.1.9\hspace{0.2cm}Require表达式}

当需要编译时谓词时,require表达式也可以作为独立的特性使用。因此,require表达式的用例可以是require子句、[static\_assert]或 \href{https://en.cppreference.com/w/cpp/language/if}{constexpr if}。

\hspace*{\fill} \\ %插入空行
\noindent
\textbf{4.1.9.1\hspace{0.2cm}static\_assert}

static\_assert需要一个编译时谓词,并在编译时谓词失败时显示一条信息。C++17中,信息是可选的。C++20中,这个编译谓词可以是require表达式。

\begin{lstlisting}[style=styleCXX]
// staticAssertRequires.cpp

#include <concepts>
#include <iostream>

struct Fir {
	int count() const {
		return 2020;
	}
};

struct Sec {
	int size() const {
		return 2021;
	}
};

int main() {

	std::cout << '\n';
	
	First first;
	static_assert(requires(Fir fir){ { fir.count() } -> std::convertible_to<int>; });
	
	Second second;
	static_assert(requires(Sec sec){ { sec.count() } -> std::convertible_to<int>; });
	
	int third;
	static_assert(requires(int third){ { third.count() } -> std::convertible_to<int>; });
	
	std::cout << '\n';

}
\end{lstlisting}

require表达式(第23、26和29行)检查对象是否有成员函数count,以及结果是否可转换为int。这个检查只对Fir类有效(第6-10行)。相反,第26行和第29行的检查会失败。

\begin{center}
\includegraphics[width=1.0\textwidth]{content/3/chapter4/images/1-9.png}\\
require表达式可作为static\_assert的谓词
\end{center}

可能会希望根据编译时检查,来编译不同的代码。这时,若将constexpr与require表达式结合使用,则可以事半功倍。

\hspace*{\fill} \\ %插入空行
\noindent
\textbf{4.1.9.2\hspace{0.2cm}constexpr if}

constexpr if在C++17中允许有条件地编译源代码。条件部分中,require表达式就可以发挥作用了,并且if的所有分支都必须有效。

得益于constexpr if,现在可以定义在编译时检查参数的函数,并根据分析生成的功能。

\begin{lstlisting}[style=styleCXX]
// constexprIfRequires.cpp
#include <concepts>
#include <iostream>

struct First {
	int count() const {
		return 2020;
	}
};

struct Second {
	int size() const {
		return 2021;
	}
};

template <typename T>
int getNumberOfElements(T t) {

	if constexpr (requires(T t){ { t.count() } -> std::convertible_to<int>; }) {
		return t.count();
	}
	if constexpr (requires(T t){ { t.size() } -> std::convertible_to<int>; }) {
		return t.size();
	}
	else return 42;

}

int main() {
	std::cout << '\n';
	
	First first;
	std::cout << "getNumberOfElements(first): " << getNumberOfElements(first) << '\n';
	
	Second second;
	std::cout << "getNumberOfElements(second): " << getNumberOfElements(second) << '\n';
	
	int third;
	std::cout << "getNumberOfElements(third): " << getNumberOfElements(i) << '\n';
	
	std::cout << '\n';

}
\end{lstlisting}

这个示例中,第21和22行非常重要。第21行中,require表达式确定变量t是否有成员函数count,该函数返回int。因此,第22行确定变量t是否具有成员函数size。

\begin{tcblisting}{commandshell={}}
getNumberOfElements(first): 2020
getNumberOfElements(second): 2021
getNumberOfElements(third): 42
\end{tcblisting}

\hspace*{\fill} \\ %插入空行
\noindent
\textbf{4.1.9.3\hspace{0.2cm}requires requires或匿名概念}

可以定义一个匿名概念并直接使用,但通常不应该这样做。匿名概念会使代码难以阅读,并且无法重用定义的概念。

\hspace*{\fill} \\ %插入空行
\noindent
\textbf{一个用于添加两个概念的匿名概念}
\begin{lstlisting}[style=styleCXX]
template<typename T>
	requires requires (T x) { x + x; }
T add1(T a, T b) { return a + b; }
\end{lstlisting}

函数模板定义了它的概念。add1在require子句中使用require表达式,匿名概念等同于前面定义的概念Addable。下面使用命名概念Addable定义的函数模板add2具有与add1相同的功能。

\hspace*{\fill} \\ %插入空行
\noindent
\textbf{Addable概念的使用}
\begin{lstlisting}[style=styleCXX]
template<Addable T>
T add2(T a, T b) { return a + b; }
\end{lstlisting}

概念应该对通用的需求进行封装,并为其提供自解释的名称以便重用,因此其对于维护代码非常重要。而匿名概念读起来,更像是模板参数上的语法约束。

\subsubsubsection{4.1.10\hspace{0.2cm}自定义概念}

前面的章节中,回答了关于概念的两个基本问题:“如何使用概念?“和“如何定义你的概念?”

本节中,将应用这些理论知识来定义更高级的概念,如Ordering、SemiRegular和Regular。

\hspace*{\fill} \\ %插入空行
\noindent
\textbf{4.1.10.1\hspace{0.2cm}Equal和Ordering的概念}

我已经在Haskell的类型的层次结构中介绍了概念的漫长历史:

\begin{center}
\includegraphics[width=0.6\textwidth]{content/3/chapter4/images/9.png}\\
Haskell类型的层次结构
\end{center}

类层次结构表明类型Ord是类型Eq的细化,Haskell优雅地表达了这一点。

\hspace*{\fill} \\ %插入空行
\noindent
\textbf{Haskell类型层次结构的一部分}
\begin{lstlisting}[style=styleCXX]
class Eq a where
	(==) :: a -> a -> Bool
	(/=) :: a -> a -> Bool

class Eq a => Ord a where
	compare :: a -> a -> Ordering
	(<) :: a -> a -> Bool
	(<=) :: a -> a -> Bool
	(>) :: a -> a -> Bool
	(>=) :: a -> a -> Bool
	max :: a -> a -> a
\end{lstlisting}

每个类型a支持类型类Eq(第1行),必须支持等式(第2行)和不等式(第3行)。支持类型类Ord的每个类型a都必须支持类型类Eq(在第5行中,(Eq a => Ord a)。此外,类型a必须支持四个比较操作符,以及compare和max函数(第6-11行)。

能否用C++20中的概念来表达Haskell在类型Eq和Ord之间的关系?为了简单起见,这里忽略Haskell的compare和max函数。

\hspace*{\fill} \\ %插入空行
\noindent
\textbf{4.1.10.1.1\hspace{0.2cm}排序概念}

有了需求表达式,排序概念的定义看起来与Haskell中类型类ord的定义相似。

\begin{lstlisting}[style=styleCXX]
template <typename T>
concept Ordering =
	Equal<T> &&
	requires(T a, T b) {
		{ a <= b } -> std::convertible_to<bool>;
		{ a < b } -> std::convertible_to<bool>;
		{ a > b } -> std::convertible_to<bool>;
		{ a >= b } -> std::convertible_to<bool>;
	};
\end{lstlisting}

排序概念在底层使用了嵌套的需求。类型T若支持Equal概念,则支持排序概念,此外还支持四个比较运算符。

\hspace*{\fill} \\ %插入空行
\noindent
\textbf{排序概念的定义和使用}
\begin{lstlisting}[style=styleCXX]
// conceptsDefinitionOrdering.cpp

#include <concepts>
#include <iostream>
#include <unordered_set>

template<typename T>
concept Equal =
	requires(T a, T b) {
		{ a == b } -> std::convertible_to<bool>;
		{ a != b } -> std::convertible_to<bool>;
	};


template <typename T>
concept Ordering =
	Equal<T> &&
	requires(T a, T b) {
		{ a <= b } -> std::convertible_to<bool>;
		{ a < b } -> std::convertible_to<bool>;
		{ a > b } -> std::convertible_to<bool>;
		{ a >= b } -> std::convertible_to<bool>;
	};

template <Equal T>
bool areEqual(const T& a, const T& b) {
	return a == b;
}

template <Ordering T>
T getSmaller(const T& a, const T& b) {
	return (a < b) ? a : b;
}

int main() {

	std::cout << std::boolalpha << '\n';

	std::cout << "areEqual(1, 5): " << areEqual(1, 5) << '\n';

	std::cout << "getSmaller(1, 5): " << getSmaller(1, 5) << '\n';

	std::unordered_set<int> firSet{1, 2, 3, 4, 5};
	std::unordered_set<int> secSet{5, 4, 3, 2, 1};

	std::cout << "areEqual(firSet, secSet): " << areEqual(firSet, secSet) << '\n';

	// auto smallerSet = getSmaller(firSet, secSet);

	std::cout << '\n';

}
\end{lstlisting}

函数模板areEqual(第25行)要求实参a和b具有相同的类型并支持Equal概念,函数模板getSmaller(第30行)要求两个参数都支持有序概念。当然,像1和5这样的整数可以同时匹配这两个概念,而\href{https://en.cppreference.com/w/cpp/container/unordered_set}{std::unordered\_set}并不匹配有序概念。因此,我将第48行注释掉。

\begin{tcblisting}{commandshell={}}
areEqual(1, 5): false
getSmaller(1, 5): 1
areEqual(firSet, secSet): true
\end{tcblisting}

接下来,当编译第48行会发生什么?GCC编译器明确指出std::unordered\_set不是函数模板getSmaller的有效参数。

\begin{center}
\includegraphics[width=1.0\textwidth]{content/3/chapter4/images/1-5.png}\\
函数模板getsmall的使用错误
\end{center}

排序概念已经是C++20标准的一部分。

\begin{itemize}
\item
std::three\_way\_comparable:等价于上面提出的排序概念

\item
std::three\_way\_comparable\_with:允许比较不同类型的值;例如:1.0 < 1.0f
\end{itemize}

C++20中,可以使用三向比较操作符,也称为宇宙飞船操作符<=>。我将在三向比较运算符的章节中介绍它。

\hspace*{\fill} \\ %插入空行
\noindent
\textbf{4.1.10.2\hspace{0.2cm}半常规(SemiRegular)和常规(Regular)的概念}

想要在C++生态系统中定义一个工作良好的类型时,应该定义一个“行为像int型”的类型。形式上,具体类型应该是常规类型在本节中,来定义半常规和常规概念。

半常规和常规是C++中的基本思想。抱歉,我应该说概念。例如,在《C++核心指南》中的解释\href{http://isocpp.github.io/CppCoreGuidelines/CppCoreGuidelines#Rt-regular}{T.46:要求模板参数至少是半常规和常规类型}。现在,只剩下一个重要的问题需要回答:什么是常规类型或半常规类型?在讨论之前,先给出结论:

\begin{itemize}
\item
常规类型“行为类似于int型”,可以复制,并且复制操作的结果独立于原始操作,具有相同的值。
\end{itemize}

更正式一点。常规类型也是半常规类型,让我们开始吧。


\begin{tcolorbox}[breakable,enhanced jigsaw,colback=blue!5!white,colframe=blue!75!black,title={常规类型}]

\href{https://en.wikipedia.org/wiki/Alexander_Stepanov}{Alexander Stepanov},标准模板库的设计者,定义了术语常规类型和半常规类型。根据他的说法,若一个类型支持这些函数,那么它就是常规类型

\begin{itemize}
\item
复制构造

\item
赋值

\item
等式

\item
析构

\item
全序
\end{itemize}

复制构造意味着默认构造,等式构造意味着不相等。Stepanov定义上述需求时,C++中还没有移动语义。Alexander Stepanov和\href{https://www.mcjones.org/paul/}{Paul McJones}合著的书\href{http://elementsofprogramming.com/}{Elements of Programming}专门介绍了常规类型。

\end{tcolorbox}

\hspace*{\fill} \\ %插入空行
\noindent
\textbf{4.1.10.2.1\hspace{0.2cm}半常规概念}

一个半常规的X型必须支持六大函数,并且可交换。六大函数包括:

\begin{itemize}
\item
默认构造函数:
\begin{lstlisting}[style=styleCXX]
X()
\end{lstlisting}

\item
复制构造函数:
\begin{lstlisting}[style=styleCXX]
X(const X&)
\end{lstlisting}

\item
复制赋值操作符:
\begin{lstlisting}[style=styleCXX]
X& operator = (const X&)
\end{lstlisting}

\item
移动构造函数:
\begin{lstlisting}[style=styleCXX]
X(X&&)
\end{lstlisting}

\item
移动赋值操作符:
\begin{lstlisting}[style=styleCXX]
X& operator = (X&&)
\end{lstlisting}

\item
析构函数:
\begin{lstlisting}[style=styleCXX]
~X()
\end{lstlisting}
\end{itemize}

此外,X必须是可交换的: swap(X\&, X\&)

\href{https://en.cppreference.com/w/cpp/header/type_traits}{类型特性库}中定义了相应概念。这里,我定义了类型特征isSemiRegular,然后用它来定义概念SemiRegular。

\begin{lstlisting}[style=styleCXX]
template<typename T>
struct isSemiRegular: std::integral_constant<bool,
								std::is_default_constructible<T>::value &&
								std::is_copy_constructible<T>::value &&
								std::is_copy_assignable<T>::value &&
								std::is_move_constructible<T>::value &&
								std::is_move_assignable<T>::value &&
								std::is_destructible<T>::value &&
								std::is_swappable<T>::value >{};


template<typename T>
concept SemiRegular = isSemiRegular<T>::value;
\end{lstlisting}

类型特性isSemiRegular(第1行)是当所有的类型特性到六大函数(第3-8行)和类型特质std::is\_swappable(第9行)都满足时实现的。定义SemiRegular概念的剩余步骤,就是使用类型特征isSemiRegular(第13行)。

我们继续解读Regular概念。

\hspace*{\fill} \\ %插入空行
\noindent
\textbf{4.1.10.2.2\hspace{0.2cm}常规概念}

我们已经完成了对Regular概念的定义。除了SemiRegular概念的需求外,Regular概念还要求类型具有相等的可比性。我已经在需求表达式一节中定义了Equal概念。因此,已经完成了,只需要把Equal和SemiRegular这两个概念组合起来就可以了。

\begin{lstlisting}[style=styleCXX]
template<typename T>
concept Regular = Equal<T> &&
					SemiRegular<T>;
\end{lstlisting}

现在,如何在C++20中定义相应的概念std::semiregular和std::regular?

\hspace*{\fill} \\ %插入空行
\noindent
\textbf{4.1.10.2.3\hspace{0.2cm}std::semiregular和std::regular}

C++20使用了现有类型特征和概念,定义了std::semiregular和std::regular概念。

\begin{lstlisting}[style=styleCXX]
template<class T>
concept movable = is_object_v<T> && move_constructible<T> &&
					assignable_from<T&, T> && swappable<T>;

template<class T>
concept copyable = copy_constructible<T> && movable<T> &&
					assignable_from<T&, T&> &&
					assignable_from<T&, const T&> && assignable_from<T&, const T>;

template<class T>
concept semiregular = copyable<T> && default_initializable<T>;

template<class T>
concept regular = semiregular<T> && equality_comparable<T>;
\end{lstlisting}

std::regular概念的定义类似于Regular概念,std::semiregular概念可以与基本概念相结合,如std::copyable和std::moveable。std::movable概念基于类型特征函数\href{https://en.cppreference.com/w/cpp/types/is_object}{std::is\_object}。cppreference.com还提供了编译时谓词的可能实现。

\begin{lstlisting}[style=styleCXX]
template< class T>
struct is_object : std::integral_constant<bool,
					std::is_scalar<T>::value ||
					std::is_array<T>::value ||
					std::is_union<T>::value ||
					std::is_class<T>::value> {};
\end{lstlisting}

若类型是标量、数组、联合体或类,则它是对象。

结束本节之前,我想使用用户定义的概念Regular和C++20的概念std::regular。regularSemiRegular.cpp完成了这项工作。

\hspace*{\fill} \\ %插入空行
\noindent
\textbf{概念Regular和SemiRegular的应用}
\begin{lstlisting}[style=styleCXX]
// regularSemiRegular.cpp

#include <concepts>
#include <vector>
#include <type_traits>

template<typename T>
struct isSemiRegular: std::integral_constant<bool,
		std::is_default_constructible<T>::value &&
		std::is_copy_constructible<T>::value &&
		std::is_copy_assignable<T>::value &&
		std::is_move_constructible<T>::value &&
		std::is_move_assignable<T>::value &&
		std::is_destructible<T>::value &&
		std::is_swappable<T>::value >{};

template<typename T>
concept SemiRegular = isSemiRegular<T>::value;

template<typename T>
concept Equal =
	requires(T a, T b) {
		{ a == b } -> std::convertible_to<bool>;
		{ a != b } -> std::convertible_to<bool>;
};

template<typename T>
concept Regular = Equal<T> &&
				SemiRegular<T>;

template <Regular T>
void behavesLikeAnInt(T) {
	// ...
}

template <std::regular T>
void behavesLikeAnInt2(T) {
	// ...
}

struct EqualityComparable { };
bool operator == (EqualityComparable const&,
				  EqualityComparable const&) {
	return true;
}

struct NotEqualityComparable { };

int main() {

	int myInt{};
	behavesLikeAnInt(myInt);
	behavesLikeAnInt2(myInt);

	std::vector<int> myVec{};
	behavesLikeAnInt(myVec);
	behavesLikeAnInt2(myVec);

	EqualityComparable equComp;
	behavesLikeAnInt(equComp);
	behavesLikeAnInt2(equComp);

	NotEqualityComparable notEquComp;
	behavesLikeAnInt(notEquComp);
	behavesLikeAnInt2(notEquComp);

}
\end{lstlisting}

我将前面代码片段中的所有部分放在一起定义Regular概念(第27行)。函数模板behavesLikeAnInt(第31行)和behavesLikeAnInt2(第36行)检查参数是否“行为像int”。使用用户自定义的概念Regular和C+20的概念std::regular来建立条件。所以,类型EqualityComparable(第41行)支持相==操作符,但类型NotEqualityComparable(第47行)不支持。在两个函数中(第64行和第65行)使用NotEqualityComparable类型是这个程序中最有趣的部分。

虽然目前处于概念实现的早期阶段,这里就比较一下新的GCC和MSVC编译器的错误消息。

\begin{itemize}
\item
GCC

我在\href{https://godbolt.org/}{Compiler Explorer}上使用当前的GCC 10.2命令行参数-std=c++20。当使用用户自定义的Regular(第64行)概念时,会出现编译错误,这里只展示比较重要的错误消息:

\begin{center}
\includegraphics[width=1.0\textwidth]{content/3/chapter4/images/1-6.png}\\
\end{center}

C++20概念std::regular更加全面。因此,第65行中的调用给出了一个更全面的错误消息:

\begin{center}
\includegraphics[width=1.0\textwidth]{content/3/chapter4/images/1-7.png}\\
\end{center}

\item
MSVC

MSVC编译器给出的错误信息就不太具体了。

\begin{center}
\includegraphics[width=1.0\textwidth]{content/3/chapter4/images/10.png}\\
使用std::regular概念时出现错误消息
\end{center}

从截图中看到的,编译器为19.27.29112的x64版本,并且使用了命令行参数/EHSC和/std:c++latest。

\end{itemize}

\hspace*{\fill} \\ %插入空行
\begin{tcolorbox}[breakable,enhanced jigsaw,colback=blue!5!white,colframe=blue!75!black,title={常规类型}]

这里我想表达一下我自己的观点。首先,我陈述事实,然后得出结论。这些事实都是基于本章所述内容。那么,哪些论点支持改进,哪些论点支持革命呢?

\hspace*{\fill} \\ %插入空行
\noindent
\textbf{改进派}

\begin{itemize}
\item
概念促进在更高抽象级别上使用泛型代码。

\item
当编译模板失败时,概念会给出可以理解的错误消息,提供了无法通过\href{https://en.cppreference.com/w/cpp/header/type_traits}{type-traits库}、\href{https://en.cppreference.com/w/cpp/language/sfinae}{SFINAE}和\href{https://en.cppreference.com/w/cpp/language/static_assert}{static\_assert}实现的功能。

\item
auto是一种不受约束的占位符。C++20中,可以将概念用作有约束的占位符。

\item
C++14中,可以使用泛型Lambda定义函数模板。
\end{itemize}

\noindent
\textbf{革命派}

\begin{itemize}
\item
概念促进在更高抽象级别上使用泛型代码。

\item
概念允许我们验证模板需求。当然,也可以通过组合\href{https://en.cppreference.com/w/cpp/header/type_traits}{type-traits库}、\href{https://en.cppreference.com/w/cpp/language/sfinae}{SFINAE}和\href{https://en.cppreference.com/w/cpp/language/static_assert}{static\_assert}来实现模板参数的验证,但是这种技术太高级了,不能将其视为通用解决方案。

\item
有了简写的函数模板语法,从而模板定义得到了根本性的改进。

\item
概念表示语义类别,而不是语法约束。我们不需要像Addable这样的概念,要求类型支持加法操作符,而应该考虑数字概念,其中数字是一个语义类别,例如:相等或有序。
\end{itemize}

\hspace*{\fill} \\ %插入空行
\noindent
\textbf{我的结论}

关于概念是该改进稳步向前,还是进行革命性的飞跃,有很多争论。争端主要是因为语义分类,我是站在革命派这一边的。诸如“数”、“相等”或“排序”等概念让我想起了\href{https://en.wikipedia.org/wiki/Plato}{Plato}的思想世界。若我们现在可以在这样的语义中进行编程,这将具有革命性意义。
\end{tcolorbox}

\begin{tcolorbox}[breakable,enhanced jigsaw,colback=mygreen!5!white,colframe=mygreen!75!black,title={总结}]
\begin{itemize}
\item
定义在特定类型或类型参数上的函数或类有一组问题,概念通过对类型参数施加语义约束来解决这些问题。

\item
概念可以应用在requires子句中,约束的模板参数,或在缩写函数模板中。

\item
概念是编译时谓词,可用于各种模板。也可以重载概念,使用概念特化模板,将概念用于成员函数或可变参数模板。

\item
因为C++20和概念,不受约束占位符(auto)和受约束占位符(concept)的使用方式统一了。无论何时使用auto,都可以使用C++20中的概念。

\item
新的简化的函数模板语法,使得定义函数模板变得更加简单。

\item
定义自己的概念之前,请先研究C++20标准中丰富的预定义概念集。定义概念时,可以使用两种技术:结合概念和编译时谓词,或者使用需求表达式。
\end{itemize}
\end{tcolorbox}

\newpage

















