\begin{center}
\includegraphics[width=0.6\textwidth]{content/3/chapter5/images/32.png}\\
Cippi在上楼
\end{center}

\subsubsubsection{5.7.1\hspace{0.2cm}std::bind\_front}

std::bind\_front (Func\&\& Func, Args\&\&…Args)会为func可调用对象创建包装器。std::bind\_front可以有任意数量的参数,并将其参数绑定到前面。

\begin{tcolorbox}[breakable,enhanced jigsaw,colback=blue!5!white,colframe=blue!75!black,title={std::bind\_front与std::bind}]
	
C++11时就有了\href{https://en.cppreference.com/w/cpp/utility/functional/bind}{std::bind}和\href{https://en.cppreference.com/w/cpp/language/lambda}{Lambda表达式}。C++20中,添加了\href{https://en.cppreference.com/w/cpp/utility/functional/bind_front}{std::bind\_front}。这可能会大伙感到疑惑。std::bind自\href{https://en.wikipedia.org/wiki/C%2B%2B_Technical_Report_1}{技术报告1} (TR1)开始可用,std::bind和Lambda表达式可以用来替换std::bind\_front。此外,std::bind\_front看起来像是std::bind的姊妹,因为std::bind支持参数的重排。当然,使用std::bind\_front是有原因的:与std::bind相反,std::bind\_front会传播底层调用操作符的异常规范。
	
\end{tcolorbox}

\hspace*{\fill} \\ %插入空行
\noindent
\textbf{比较std::bind\_front、std::bind和Lambda表达式}
\begin{lstlisting}[style=styleCXX]
// bindFront.cpp

#include <functional>
#include <iostream>

int plusFunction(int a, int b) {
	return a + b;
}

auto plusLambda = [](int a, int b) {
	return a + b;
};

int main() {

	std::cout << '\n';
	
	auto twoThousandPlus1 = std::bind_front(plusFunction, 2000);
	std::cout << "twoThousandPlus1(20): " << twoThousandPlus1(20) << '\n';
	
	auto twoThousandPlus2 = std::bind_front(plusLambda, 2000);
	std::cout << "twoThousandPlus2(20): " << twoThousandPlus2(20) << '\n';
	
	auto twoThousandPlus3 = std::bind_front(std::plus<int>(), 2000);
	std::cout << "twoThousandPlus3(20): " << twoThousandPlus3(20) << '\n';
	
	std::cout << "\n\n";
	
	using namespace std::placeholders;
	
	auto twoThousandPlus4 = std::bind(plusFunction, 2000, _1);
	std::cout << "twoThousandPlus4(20): " << twoThousandPlus4(20) << '\n';
	
	auto twoThousandPlus5 = [](int b) { return plusLambda(2000, b); };
	std::cout << "twoThousandPlus5(20): " << twoThousandPlus5(20) << '\n';
	
	std::cout << '\n';

}
\end{lstlisting}

每个调用(第18、21、24、31和34行)获得一个带有两个参数的可调用对象,并返回一个只带有一个参数的可调用对象,并且第一个参数绑定到2000。可调用对象是一个函数(第18行)、一个Lambda表达式(第21行)和一个预定义的函数对象(第24行)。\_1就是占位符(第31行),代表缺失的参数。使用Lambda表达式(第34行),可以直接应用一个参数,并为缺失的形参提供参数b。从可读性的角度来看,std::bind\_front可能比std::bind或Lambda表达式更容易阅读。

\begin{tcblisting}{commandshell={}}
twoThousandPlus1(20): 2020
twoThousandPlus2(20): 2020
twoThousandPlus3(20): 2020

twoThousandPlus4(20): 2020
twoThousandPlus5(20): 2020
\end{tcblisting}

\subsubsubsection{5.7.2\hspace{0.2cm}std::is\_constant\_evaluated}

std::is\_constant\_evaluted决定函数是在编译时执行,还是在运行时执行。为什么需要类型特征库中的这个函数呢?C++20中,我们大致介绍了三种函数:

\begin{itemize}
\item 
consteval声明函数在编译时运行:consteval int alwaysCompiletime();

\item 
constexpr声明的函数可以在编译时或运行时运行;constexpr int itDepends();

\item 
一般的函数在运行时运行:int alwaysRuntime();
\end{itemize}

现在,我要写一个复杂的例子:constexpr函数可以在编译时或运行时运行。有时这些函数的行为应该不同,这取决于函数是在编译时执行,还是在运行时执行。像getSum这样的constexpr函数有可能就在编译时运行。

\hspace*{\fill} \\ %插入空行
\noindent
\textbf{使用constexr的函数声明}
\begin{lstlisting}[style=styleCXX]
constexpr int getSum(int l, int r) {
	return l + r;
}
\end{lstlisting}

如何确定函数在编译时执行?有三种可能。

\begin{enumerate}
\item 
constexpr函数在编译时执行:
\begin{itemize}
\item 
该函数用于常量求值上下文中,常量求值上下文可以在constexpr函数或static\_assert中使用。

\item 
函数显式地希望在编译时得到结果:constexpr auto res = getSum(2000,11)。现在,getSum()必须在编译时运行.
\end{itemize}

\item 
若参数不是constexpr,则只能在运行时执行constexpr函数。函数getSum(a, 11)用一个变量调用,若这个变量没有声明为constexpr: int a = 2000,就会出现这种情况。

\item 
规则1和规则2都不适用时,就可以在编译时或运行时执行constexpr函数。所以两个选项都是有效的,而这个决策则交由编译器完成。
\end{enumerate}

正是第3点,使得std::is\_constant\_evaluate的功能开始发挥作用。可以检测程序是在编译时运行,还是在运行时运行,并执行不同的操作。\href{https://en.cppreference.com/w/cpp/types/is_constant_evaluated}{cppreference.com/is\_constant\_evaluated}显示了一个不过的用例。在编译时,手动计算两个数字的幂;在运行时,使用std::pow。

\hspace*{\fill} \\ %插入空行
\noindent
\textbf{编译时和运行时执行不同的代码}
\begin{lstlisting}[style=styleCXX]
// constantEvaluated.cpp

#include <type_traits>
#include <cmath>
#include <iostream>

constexpr double power(double b, int x) {
	if (std::is_constant_evaluated() && !(b == 0.0 && x < 0)) {
		
		if (x == 0)
			return 1.0;
		double r = 1.0, p = x > 0 ? b : 1.0 / b;
		auto u = unsigned(x > 0 ? x : -x);
		while (u != 0) {
			if (u & 1) r *= p;
			u /= 2;
			p *= p;
		}
		return r;
	}
	else {
		return std::pow(b, double(x));
	}
}

int main() {
	
	std::cout << '\n';
	
	constexpr double kilo1 = power(10.0, 3);
	std::cout << "kilo1: " << kilo1 << '\n';
	
	int n = 3;
	double kilo2 = power(10.0, n);
	std::cout << "kilo2: " << kilo2 << '\n';
	
	std::cout << '\n';
}
\end{lstlisting}

我想分享一个有趣的观察结果。可以在consteval声明的函数,或只能在运行时运行的函数中使用std::is\_constant\_evaluate,这些调用的结果总是true或false。

\subsubsubsection{5.7.3\hspace{0.2cm}std::source\_location}

std::source\_location表示源代码的信息,包括文件名、行号和函数名。当需要关于调用站点的信息时,例如:用于调试、日志记录或测试目的时,这些信息非常有价值。类std::source\_location是比C++11宏\_\_FILE\_\_和\_\_LINE\_\_更好的选择,推荐使用。

std::source\_location可以给你以下信息。

\begin{center}
std::source\_location src
\end{center}

\begin{table}[H]
\centering
\begin{tabular}{ll}
\textbf{函数}                & \textbf{描述}                      \\ \hline
std::source\_location::current() & 创建一个source\_location对象src \\
src.line()                       & 返回行号                   \\
src.column()                     & 返回列号                 \\
src.file\_name()                 & 返回文件名                     \\
src.function\_name()             & 返回函数名                
\end{tabular}
\end{table}

使用std::source\_location::current()创建一个新的源位置对象src,该对象表示相应调用点的信息。2020年底,没有C++编译器支持std::source\_location,所以下面的程序sourcellocation.cpp来自\href{https://en.cppreference.com/w/cpp/utility/source_location}{cppreference.com/source\_location}。

\hspace*{\fill} \\ %插入空行
\noindent
\textbf{使用std::source\_location显示有关调用点的信息}
\begin{lstlisting}[style=styleCXX]
// sourceLocation.cpp
// from cppreference.com

#include <iostream>
#include <string_view>
#include <source_location>

void log(std::string_view message,
		 const std::source_location& location = std::source_location::current())
{
	std::cout << "info:"
			  << location.file_name() << ':'
			  << location.line() << ' '
			  << message << '\n';
}

int main()
{
	log("Hello world!"); // info:main.cpp:19 Hello world!
}
\end{lstlisting}

程序的输出是其源码的一部分。

\begin{tcolorbox}[breakable,enhanced jigsaw,colback=mygreen!5!white,colframe=mygreen!75!black,title={总结}]

\begin{itemize}
\item 
std::bind\_front是std::bind(C++11)变体,可能比std::bind更容易使用。与std::bind不同,std::bind\_front不允许对其参数进行重新排列。

\item 
std::is\_constant\_evaluted决定函数是在编译时执行,还是在运行时执行。

\item 
std::source\_location表示源代码的信息。此信息包括文件名、行号和函数名,对于调试、日志记录或测试非常有用。
\end{itemize}

\end{tcolorbox}








