\begin{center}
\includegraphics[width=0.6\textwidth]{content/3/chapter5/images/3.png}\\
Cippi在遛狗
\end{center}

std::span表示一个对象,该对象引用一个连续的对象序列。std::span,有时也称为视图,不是底层数据的所有者。这个连续的对象序列可以是一个普通的C数组、有长度信息的指针、std::array、std::vector或std::string。

std::span可以具有静态范围或动态范围。默认情况下,std::span有一个动态范围:

\hspace*{\fill} \\ %插入空行
\noindent
\textbf{std::span的定义}
\begin{lstlisting}[style=styleCXX]
template <typename T, std::size_t Extent = std::dynamic_extent>
class span;
\end{lstlisting}

\subsubsubsection{5.2.1\hspace{0.2cm}静态范围与动态范围}

当std::span有一个静态范围时,其大小在编译时已知,并且是类型:std::span<T, size>的一部分。因此,其实现只需要一个指向连续对象序列的第一个元素的指针。

具有动态范围的std::span,由指向第一个元素的指针和连续对象序列的长度组成。长度信息不是类型std::span<T>的一部分。

staticDynamicExtentSpan.cpp展示了这两种视图之间的区别。

\begin{lstlisting}[style=styleCXX]
// staticDynamicExtentSpan.cpp

#include <iostream>
 #include <span>
 #include <vector>

void printMe(std::span<int> container) {
	
	std::cout << "container.size(): " << container.size() << '\n';
	for (auto e : container) std::cout << e << ' ';
	std::cout << "\n\n";
}

int main() {

	std::cout << '\n';
	
	std::vector myVec1{1, 2, 3, 4, 5};
	std::vector myVec2{6, 7, 8, 9};
	
	std::span<int> dynamicSpan(myVec1);
	std::span<int, 4> staticSpan(myVec2);
	
	printMe(dynamicSpan);
	printMe(staticSpan); // implicitly converted into a dynamic span
	
	// staticSpan = dynamicSpan; ERROR
	dynamicSpan = staticSpan;
	
	printMe(staticSpan);
	
	std::cout << '\n';

}
\end{lstlisting}

dynamicSpan(第21行)有一个动态范围,而staticSpan(第22行)有一个静态范围。两个std::span都在printMe函数中返回它们的长度(第9行)。具有动态范围的std::span可以赋值给具有静态范围的std::span,但不能反过来。第27行会导致错误,而第7、25和28行是可以工作的。

\begin{center}
\includegraphics[width=0.5\textwidth]{content/3/chapter5/images/4.png}\\
std::span的静态范围与动态范围
\end{center}

使用std::span<T>的一个重要原因是,若将普通C数组\href{https://en.cppreference.com/w/cpp/types/decay}{退化}成指针,那么长度信息就会丢失。这种退化是C/C++中出现错误的原因之一。

\subsubsubsection{5.2.2\hspace{0.2cm} Automatically Deduces the Size of a Contiguous Sequence of Objects}

In contrast to a C-array, std::span<T> automatically deduces the size of contiguous sequences of objects.

\begin{lstlisting}[style=styleCXX]
// printSpan.cpp

#include <iostream>
#include <vector>
#include <array>
#include <span>

void printMe(std::span<int> container) {

	std::cout << "container.size(): " << container.size() << '\n';
	for (auto e : container) std::cout << e << ' ';
	std::cout << "\n\n";
	
}

int main() {

	std::cout << '\n';
	
	int arr[]{1, 2, 3, 4};
	printMe(arr);
	
	std::vector vec{1, 2, 3, 4, 5};
	printMe(vec);
	
	std::array arr2{1, 2, 3, 4, 5, 6};
	printMe(arr2);

}
\end{lstlisting}

The C-array (line 19), std::vector (line 22), and the std::array (line 25) contain int values. Consequently, std::span also holds int values. There is something more interesting in this simple example. For each container, std::span can deduce its size (line 10).

\begin{center}
\includegraphics[width=0.4\textwidth]{content/3/chapter5/images/5.png}\\
Automatic size deduction of a std::span
\end{center}

There are more ways to create a std::span.


\subsubsubsection{5.2.3\hspace{0.2cm} Create a std::span from a Pointer and a Size}

You can create a std::span from a pointer and a size.

\hspace*{\fill} \\ %插入空行
\noindent
Create a std::span
\begin{lstlisting}[style=styleCXX]
// createSpan.cpp

#include <algorithm>
#include <iostream>
#include <span>
#include <vector>

int main() {
	
	std::cout << '\n';
	std::cout << std::boolalpha;
	
	std::vector myVec{1, 2, 3, 4, 5};
	
	std::span mySpan1{myVec};
	std::span mySpan2{myVec.data(), myVec.size()};
	
	bool spansEqual = std::equal(mySpan1.begin(), mySpan1.end(),
	                             mySpan2.begin(), mySpan2.end());
	
	std::cout << "mySpan1 == mySpan2: " << spansEqual << '\n';
	
	std::cout << '\n';

}
\end{lstlisting}

As you may expect, mySpan1, created from the std::vector (line 15), and mySpan2, created from a pointer and a size (line 16), are equal (line 21).

\begin{center}
\includegraphics[width=0.6\textwidth]{content/3/chapter5/images/6.png}\\
Create a std::span from a pointer and a size
\end{center}

\begin{tcolorbox}[breakable,enhanced jigsaw,colback=red!5!white,colframe=red!75!black,title={A std::span is neither a std::string\_view nor a view}]
You may remember that a std::span is sometimes called a view. Don’t confuse a std::span with a view from the ranges library or a \href{https://www.modernescpp.com/index.php/c-17-what-s-new-in-the-library}{std::string\_view}.

A view from the ranges library is something that you can apply on a range and performs some operation. A view does not own data, and its time for each copy, move, and assignment is constant. A std::span and a std::string\_view are non-owning views and can deal with strings.

The main difference between a std::span and a std::string\_view is that a std::span can modify its referenced objects.
\end{tcolorbox}
	
\subsubsubsection{5.2.4\hspace{0.2cm} Modifying the Referenced Objects}

You can modify an entire span or only a subspan. When you modify a span, you modify the referenced objects.

The following program shows how a subspan can be used to modify the referenced objects from a std::vector.

\hspace*{\fill} \\ %插入空行
\noindent
Modify the objects referenced by a std::span
\begin{lstlisting}[style=styleCXX]
// spanTransform.cpp

#include <algorithm>
 #include <iostream>
 #include <vector>
 #include <span>

void printMe(std::span<int> container) {
	
	 std::cout << "container.size(): " << container.size() << '\n';
	 for (auto e : container) std::cout << e << ' ';
	 std::cout << "\n\n";
}

int main() {

	std::cout << '\n';
	
	std::vector vec{1, 2, 3, 4, 5, 6, 7, 8, 9, 10};
	printMe(vec);
	
	std::span span1(vec);
	std::span span2{span1.subspan(1, span1.size() - 2)};
	
	
	std::transform(span2.begin(), span2.end(),
	               span2.begin(),
	               [](int i){ return i * i; });
	
	
	printMe(vec);
	printMe(span1);

}
\end{lstlisting}

span1 references the std::vector vec (line 22). In contrast, span2 references only the elements of the underlying vec excluding the first and the last element (line 23). Consequently, the mapping of each element to its square (line 26) only addresses these elements.

\begin{center}
\includegraphics[width=0.6\textwidth]{content/3/chapter5/images/7.png}\\
Modify the objects referend by a std::span
\end{center}

There are various convenience functions to address the elements of the std::span.

\subsubsubsection{5.2.5\hspace{0.2cm} Addressing std::span Elements}

The following table presents the functions to refer to the elements of a std::span.

\begin{center}
Interface of a std::span spse
\end{center}

\begin{table}[H]
\centering
\begin{tabular}{ll}
Function         & Description                                         \\ \hline
sp.front()       & Access the first element.                           \\
sp.back()        & Access the last element.                            \\
sp{[}i{]}        & Access the i-th element.                            \\
sp.data()        & Returns a pointer to the beginning of the sequence. \\
sp.size()        & Returns the number of elements of the sequence.     \\
sp.size\_bytes() & Returns the size of the sequence in bytes.          \\
sp.empty()       & Returns true if the sequence is empty.              \\
\begin{tabular}[c]{@{}l@{}}sp.first\textless{}count\textgreater{}()\\ sp.frist(count)\end{tabular} &
Returns a subspan consisting of the first count elements of the sequence. \\
\begin{tabular}[c]{@{}l@{}}sp.last\textless{}count\textgreater{}()\\ sp.last(count)\end{tabular} &
Returns a subspan consisting of the last count elements of the sequence. \\
\begin{tabular}[c]{@{}l@{}}sp.subspan\textless{}first, count\textgreater{}()\\ sp.subspan(first, count)\end{tabular} &
Returns a subspan consisting of count elements starting at first.
\end{tabular}
\end{table}

The program subspan.cpp shows the usage of the member function subspan.

\hspace*{\fill} \\ %插入空行
\noindent
Use of the member function subspan
\begin{lstlisting}[style=styleCXX]
// subspan.cpp

#include <iostream>
#include <numeric>
#include <span>
#include <vector>

int main() {

	std::cout << '\n';
	
	std::vector<int> myVec(20);
	std::iota(myVec.begin(), myVec.end(), 0);
	for (auto v: myVec) std::cout << v << " ";
	
	std::cout << "\n\n";
	
	std::span<int> mySpan(myVec);
	auto length = mySpan.size();
	
	std::size_t count = 5;
	for (std::size_t first = 0; first <= (length - count); first += count ) {
		for (auto ele: mySpan.subspan(first, count)) std::cout << ele << " ";
		std::cout << '\n';
	}

}
\end{lstlisting}

Line 13 fills the vector with all numbers from 0 to 19 (line 13) using the algorithm \href{https://en.cppreference.com/w/cpp/algorithm/iota}{std::iota}. This vector is further used to initialize a std::span (line 18). Finally, the for loop (line 22) uses the function subspan to create all subspans starting at first and having count elements until mySpan is consumed.

\begin{center}
\includegraphics[width=0.6\textwidth]{content/3/chapter5/images/8.png}\\
Use of the member function subspan
\end{center}

Kilian Henneberger reminded me of a special use case of std::span. A constant range of modifiable elements.

\subsubsubsection{5.2.6\hspace{0.2cm} A Constant Range of Modifiable Elements}

For simplicity, I name a std::vector and a std::span a range. A std::vector, like a std::string models a modifiable range of modifiable elements: std::vector<T>. When you declare this std::vector as const, the range models a constant range of constant objects: const std::vector<T>. You cannot model a constant range of modifiable elements. Here comes std::span into play. A std::span models a constant range of modifiable objects: std::span<T>. The following table emphasizes the variations of (constant/modifiable) ranges and (constant/modifiable) elements.

\begin{center}
(Constant/modifiable) ranges of (constant/modifiable) elements
\end{center}

\begin{table}[H]
\centering
\begin{tabular}{lll}
&
\textbf{Modifiable Elements} &
\textbf{Constant Elements} \\
\textbf{Modifiable Ranges} &
std::vector\textless{}T\textgreater{} &
\\
\textbf{Constant Ranges} &
std::span\textless{}T\textgreater{} &
\begin{tabular}[c]{@{}l@{}}const std::vector\textless{}T\textgreater\\ std::span\textless{}const T\textgreater{}\end{tabular}
\end{tabular}
\end{table}

The program constRangeModifiableElements.cpp exemplifies each combination.

\hspace*{\fill} \\ %插入空行
\noindent
(Constant/modifiable) ranges of (constant/modifiable) elements
\begin{lstlisting}[style=styleCXX]
// constRangeModifiableElements.cpp

#include <iostream>
#include <span>
#include <vector>

void printMe(std::span<int> container) {

	std::cout << "container.size(): " << container.size() << '\n';
	for (auto e : container) std::cout << e << ' ';
	std::cout << "\n\n";
}

int main() {

	std::cout << '\n';
	
	std::vector<int> origVec{1, 2, 2, 4, 5};
	
	// Modifiable range of modifiable elements
	std::vector<int> dynamVec = origVec;
	dynamVec[2] = 3;
	dynamVec.push_back(6);
	printMe(dynamVec);
	
	// Constant range of constant elements
	const std::vector<int> constVec = origVec;
	// constVec[2] = 3; ERROR
	// constVec.push_back(6); ERROR
	std::span<const int> constSpan(origVec);
	// constSpan[2] = 3; ERROR
	
	// Constant range of modifiable elements
	std::span<int> dynamSpan{origVec};
	dynamSpan[2] = 3;
	printMe(dynamSpan);
	
	std::cout << '\n';

}
\end{lstlisting}

The vector dynamVec (line 21) is a modifiable range of modifiable elements. This observation does not hold for the vector constVec (line 27). Neither can constVec change an element nor its size. constSpan (line 30) behaves accordingly. dynamSpan models the unique use case of a constant range of modifiable elements.

\begin{center}
\includegraphics[width=0.6\textwidth]{content/3/chapter5/images/9.png}\\
(Constant/modifiable) ranges of (constant/modifiable) elements
\end{center}

\begin{tcolorbox}[breakable,enhanced jigsaw,colback=mygreen!5!white,colframe=mygreen!75!black,title={Distilled Information}]

\begin{itemize}
\item 
A std::span is an object that refers to a contiguous sequence of objects. A std::span, also known as view, is never an owner and, therefore, does not allocate memory. The contiguous sequence of objects can be a plain C-array, a pointer with a size, a std::array, a std::vector, or a std::string.

\item 
In contrast to a C-array, a std::span automatically deduces the size of its referenced sequence of objects.

\item 
When a std::span modifies its elements, the reference objects are also modified.
\end{itemize}

\end{tcolorbox}










